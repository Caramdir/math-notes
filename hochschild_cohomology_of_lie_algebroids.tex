%type: notes
%title: Hochschild cohomology of Lie algebroids
%tags: Hochschild cohomology, Lie algebroids
\documentclass[english,no-theorem-numbers]{short-notes}

\usepackage{math-alg,math-ag}

\addbibresource{math.bib}
%\bibliography{math.bib}

\title{Hochschild cohomology of Lie algebroids}
\author{Clemens Koppensteiner}

\begin{document}

\maketitle

These notes are mainly about \cite{CalaqueRosseVanDenBergh:2010:HochschildCohomolgyOfLieAlgebroids}.

\section*{An example}

We do explicit computations of the simplest possible example, $X = \as 1$ with $\sheaf L = \sheaf T_X$.
Thus
\begin{itemize}
    \item $\O_X = ℂ[x]$.
    \item $\sheaf L = ℂ[x]∂_x$.
    \item $U_X\sheaf L = D_X = ℂ[x][∂_x]$ with $[∂_x,x] = 1$.
\end{itemize}
Because of the relation $[∂_x,x]=1$, a map of $\O_X$-modules $U_X\sheaf L → \O_X$ is given by the images of $∂_x^n$ (including $∂_x^0 = 1$).
Hence $J_X\sheaf L = ℂ[x]\lBrack dx \rBrack$ (with $dx(∂_x) = 1$).
\newcommand\JXL{ℂ[x]\lBrack dx \rBrack}%

% algebra structure.

The $J_X\sheaf L$-modules structure on $\O_X$ is given via $J_X\sheaf L → \O_X$, $φ\mapsto φ(1)$ (i.e.~mapping a power series to its $0$th coefficient).

Hochschild cohomology of $\sheaf L$ is then the cohomology groups of
\[
    R\Hom_{\JXL}(ℂ[x],ℂ[x]).
\]
The obvious resolution for calculating this is
\[
    \left(\JXL \xrightarrow{\cdot dx} \JXL \right) → ℂ[x].
\]
Hence,
\[
    R\Hom_{\JXL}(ℂ[x], ℂ[x]) = \left( ℂ[x] \xrightarrow{0} ℂ[x] \right) =
    \SymAlg_{ℂ[x]}(\sheaf T_X[1]) = 
    \operatorname{HC}^\cx(X).
\]

\section*{Questions}

\begin{enumerate}
    \item
        If $\sheaf L$ is not locally free (i.e.\ not a bundle), does the construction still work.
        In particular, if $X$ is a symplectic variety and $\sheaf L$ is the Poisson Lie algebroid or the $\O_X$-module generated by Hamiltonian vector fields, what do we get?
        (Possibly the next question is of use for this too.)
    \item
        If $X$ is singular, is there a derived version of the construction (possibly restricting to the case that $\sheaf L$ is perfect)?
        If so, what do we get when $\sheaf L$ is the tangent complex?
    \item
        Is the assignment $\sheaf L \mapsto \operatorname{HH}_{\sheaf L}^{\cx}$ functorial?
\end{enumerate}

\section{Groupoids and algebroids}

In \cite[Section~F.4]{ArinkinGaitsgory:arXiv:v2:SingularSupport}, the authors construct an $\mathbb E₂$-algebra $\mathcal A_{\mathcal G}$ associated to a groupoid $\mathcal G$ over $Z$.

\begin{Question}
    Let $\sheaf L$ be the Lie algebroid associated to the groupoid $\mathcal G$.
    Is the Hochschild cohomology of $\sheaf L$ the same as $\mathcal A_{\mathcal G}$ (at least on the level of cohomology)?
\end{Question}

\begin{enumerate}
    \item Assume that $\sheaf L$ is a vector bundle (and $Z$ smooth).
        \begin{enumerate}
            \item
                Can we see the equivalence directly from the definition?
                For example, is $J_Z\sheaf L \cong \O_{\mathcal G}$?
                (For the pair groupoid this is contained in \cite[Proposition~6.1]{CalaqueRossiVanDenBergh:2010:HochschildCoHomlogyForLieAlgebroids}.)
            \item
                Is the Lie algebra in \cite[Proposition G.1.7]{ArinkinGaitsgory:arXiv:v2:SingularSupport} the same as $\sheaf L[-1]$ with the bracket given by the Atiyah class of $\sheaf L^*$?
            \item
                Does the Lie bracket on $\sheaf L$ extend to give a Gerstenhaber bracket on $U(\sheaf L[-1])$?
                Is this the same Gerstenhaber bracket as the one induced by the $\mathbb E₂$-structure on $\mathcal A_{\mathcal G}$?
                How does this compare (via HKR) with $\SymAlg \sheaf L[-1] = \bigwedge^\ast \sheaf L$?
        \end{enumerate}
    \item Can these concepts be generalized to the non-smooth setting, i.e.~where $\sheaf L$ is only a perfect complex?
    \item Can we thus define singular support for arbitrary $\sheaf L$, and is this useful?
        For $D(Z)$ this would be strictly less information that the full singular support (when $\sheaf L$ is the tangent complex), but maybe the category of $\mathcal A_{\mathcal G}$- ($\mathcal A_{\sheaf L}$-)modules has a nice description in special cases?
\end{enumerate}

\printbibliography

\end{document}
