%type: notes
%title: The commuting stack
%tags: representation theory, elliptic curve, local systems, geometric langlands
\documentclass[english, no-theorem-numbers]{short-notes}

\usepackage{math-alg,math-ag,math-gl}

\addbibresource{global.bib}
%\bibliography{global.bib}


\title{the commuting stack}
\author{Clemens Koppensteiner}

\begin{document}

\maketitle

Let $G$ be a reductive algebraic group and let $E$ be an elliptic curve.
The moduli space of local $G$-systems on $E$ is called the \emph{commuting stack}:
\[
    \Loc_G(E) = \rquot{\{π₁(E) → G\}}G,
\]
where the quotient is taken via the adjoint action of $G$ on itself.
Picking a basis for $π₁(E)$, one has
\[
    \Loc_G'(E) = \{π₁(E) → G\} = \{(a,b) ∈ G : [a,b] = 1 \},
\]
i.e.\ a pullback diagram
\[
    \begin{tikzpicture}
        \matrix[commutative diagram] (m) {
            \Loc_G'(E) & G² \\
            1 & G \\
        };
        \path[commutative diagram arrows]
            (m-1-1) edge (m-1-2) edge (m-2-1)
            (m-1-2) edge node[right] {$[\,{,}\,]$} (m-2-2)
            (m-2-1) edge (m-2-2);
    \end{tikzpicture}
\]
We probably want to interpret this diagram as a diagram in derived schemes.

The commuting stack can also be described as an iterated loop space:
\[
    \Loc_G(E) = \mathcal L(\mathcal L \mathrm BG) = \mathcal L(G/G).
\]
See the discussion in \cite[Section~1.3]{BenZviNadler:2013:LoopSpacesAndRepresentations} for details.

\printbibliography
\end{document}
