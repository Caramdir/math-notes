%type: notes
%title: Braden's hyperbolic localization for ind-coherent sheaves
%tags: algebraic geometry, sheaf theory
\documentclass[english]{short-notes}

\usepackage{math-alg,math-ag, math-gl}
\usepackage{xparse}

\addbibresource{global.bib}
%\bibliography{global.bib}

\title{braden's hyperbolic localization for ind-coherent sheaves}
\author{Clemens Koppensteiner}

\newcommand\pt{\mathrm{pt}}
\newcommand\bMaps{\mathbf{Maps}}
\newcommand\catIndCohMon[2]{\catIndCoh{#1}^{#2\mathrm{-mon}}}
\let\IC\catIndCoh
\let\ICM\catIndCohMon

\begin{document}

\maketitle

This is an attempt to formulate and prove a version of Braden's hyperbolic localization theorem \cite{Braden:2003:HyperbolicLocalizationOfIC, DrinfeldGaitsgory:arXiv:OnATheoremOfBraden} for ind-coherent sheaves on a dg stack.

\begin{enumerate}
    \item Check whether the statement makes sense on schemes/algebraic spaces.
    \item Extend it to stacks.
    \item Check whether the foundations can be extended to dg-schemes/stacks.
\end{enumerate}

\section{Ind-coherent sheaves on algebraic spaces}

\subsection{The setup}

Let us recall the setup of \cite{DrinfeldGaitsgory:arXiv:OnATheoremOfBraden}.
We start with a separated algebraic space $Z$ of finite type, defined over a field $k$, and a $\Gm$-action on $Z$.
The \emph{subspace of fixed points} is
\[
    Z^0 = \bMaps^{\Gm}(\pt,Z).
\]
The \emph{attractor} and \emph{repeller} of $Z$ are
\[
    Z^+ = \bMaps^{\Gm}(\as 1,Z)
    \quad\text{and}\quad
    Z^- = \bMaps^{\Gm}(\as[-]1,Z).
\]
We have $\Gm$-equivariant morphisms 
\[ 
    p^\pm\colon Z^\pm → Z
\]
and 
\[
    i^\pm\colon Z^0 \rightleftarrows Z^\pm \cocolon q^\pm
\]
with $q^\pm∘i^\pm = \id_{Z^0}$ and $p^\pm ∘ i^\pm$ equal to the embedding $Z^0 \hookrightarrow Z$.
Further let
\[
    j = (i^+, i^-)\colon Z^0 → Z^+ ×_{Z} Z^-.
\]
To summarize, we have the following diagram (where the solid part commutes and the square is Cartesian):
\[
    \begin{tikzpicture}
        \matrix[spaced commutative diagram] (m) {
            Z^0 & & \\
            & Z^+ ×_Z Z^- & Z^+ \\
            & Z^- & Z \\
        };
        \path[commutative diagram arrows] 
            (m-1-1) edge node[above] {$i^+$} (m-2-3)
                    edge node[descr] {$j$} (m-2-2)
                    edge node[left] {$i^-$} (m-3-2)
            (m-2-2) edge node[descr] {$'p^-$} (m-2-3)
                    edge node[descr] {$'p^+$} (m-3-2)
            (m-2-3) edge node[right] {$p^+$} (m-3-3)
                    edge[dashed,bend left=25] node[descr] {$q^+$} (m-2-2)
            (m-3-2) edge node[below] {$p^-$} (m-3-3)
                    edge[dashed,bend right=25] node[descr] {$q^-$} (m-2-2);
    \end{tikzpicture}
\]
These maps and spaces have the following major properties:
\begin{enumerate}
    \item $Z^0$ and $Z^\pm$ separated algebraic space of finite type.
    \item The $\Gm$-action on $Z^0$ is trivial.
    \item If $Z$ is a scheme, then so are $Z^0$ and $Z^\pm$.
    \item If $Z$ is smooth, then so are $Z^0$ and $Z^\pm$ and $q^\pm$ is smooth.
    \item $q^\pm$ is affine.
    \item $p^\pm$ is a monomorphism.
    \item $i^\pm$ is a closed embedding.
    \item $j$ is a closed and open embedding.
    \item If $Z$ is affine, then $p^\pm$ is a closed embedding and $j$ an isomorphism.
\end{enumerate}

If $G$ is an algebraic group acting on a space $Z$ we write 
\[ 
    \catIndCohMon{Z}{G} \subseteq \catIndCoh Z
\]
for the full subcategory generated by the essential image of the pullback functor $\catIndCoh{\rquot ZG} → \catIndCoh Z$ (since the map $Z → \rquot ZG$ is smooth the $!$- and $*$-pullback functors differ only by tensoring with a cohomologically shifted line bundle).
In particular, we consider the categories $\catIndCohMon{Z}{\Gm}$, $\catIndCohMon{Z^\pm}{\Gm}$ and $\catIndCohMon{Z^0}{\Gm} = \catIndCoh{Z^0}$\todo{Verify that for trival action the categories coincide.}.
We have functors\todo{Verify that the functors actually restrict to these categories.}
\[
    (p^\pm)^!\colon \ICM{Z}{\Gm} → \ICM{Z^\pm}{\Gm}
    \quad\text{and}\quad
    (i^\pm)^!\colon \ICM{Z^\pm} → \IC{Z^0},
\]
and $(p^\pm)_*^\IndCoh$, $(i^\pm)_*^\IndCoh$.

\begin{Assumption}
    We assume that the maps $p^\pm$ and $i^\pm$ are eventually coconnective.
\end{Assumption}

In this case the maps $(p^\pm)_*^\IndCoh$, $(i^\pm)_*^\IndCoh$ have left-adjoints $(p^\pm)^*_\IndCoh$ and $(i^\pm)^*_\IndCoh$.

\subsection{The contraction principle}

For this section we set $Y = Z^+$, $Y^0 = Z^0$, $i = i^+$ and $π = q = q^+$.
The first step to the hyperbolic localization theorem is the following \enquote{contraction principle}.

\begin{Thm}
    \begin{enumerate}
        \item There is a canonical isomorphism 
            \[
                \res{i^*_{\IndCoh}}{\ICM{Y}{\Gm}} \cong \res{q_*^\IndCoh}{\ICM{Y}{\Gm}}.
            \]
            More precisely, for each $\sheaf F ∈ \ICM{Y}{\Gm}$ the natural map
            \[
                q_*(\sheaf F) →
                q_*∘i_*∘i^*(\sheaf F) →
                (q∘i)_* ∘ i^*(\sheaf F) = i^*(\sheaf F)
            \]
            is an isomorphism.
        \item The morphism $q^!$ has a left adjoint $q_!$ and there is a canonical isomorphism 
            \[
                \res{q_!}{\ICM{Y}{\Gm}} \cong \res{i^!}{\ICM{Y}{\Gm}}.
            \]
            More precisely, for each $\sheaf F ∈ \ICM{Y}{\Gm}$ the natural map
            \[
                i^!(\sheaf F) →
                i^!∘q^!∘q_!(\sheaf F) →
                (q∘i)^! ∘q_!(\sheaf F) = q_!(\sheaf F)
            \]
            is an isomorphism.
    \end{enumerate}
\end{Thm}

In other words, the theorem says that $q_*$ left-adjoint to $i_*$ with counit the equality $q_* ∘ i_* → \id_{\IC{Z^0}}$, and similarly for the second part.
The proof of this theorem should follow \cite[Appendix~C]{DrinfeldGaitsgory:arXiv:CompactGenerationOfDModOnBunG}.


\printbibliography
\end{document}
