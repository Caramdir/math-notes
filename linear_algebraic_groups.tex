%type: notes
%title: Linear algebraic groups
%tags: representation theory, algebraic groups, group theory
\documentclass[english, no-theorem-numbers]{short-notes}

\usepackage{math-alg,math-ag}

\addbibresource{global.bib}
%\bibliography{global.bib}

\newcommand\Humph[1]{\cite[#1]{Humphreys:1975:LinearAlgebraicGroups}}
\newcommand\open{\circ}
\newcommand\conn{\circ}

\title{linear algebraic groups}
\author{Clemens Koppensteiner}

\begin{document}
\maketitle

\section*{theory}

\subsection*{general definitions}

Let $G$ be an affine algebraic group over an algebraically closed field $k$ of characteristic $0$.

%\begin{Def}
%    A \emph{representation} of $G$ is an homomorphism $ρ\colon G → \operatorname{GL}(V)$ for some finite dimensional vector space $V$ over $k$.
%\end{Def}
%
%\begin{Def}
%    The \emph{character} of a representation $ρ$ is the map $\trace ∘ ρ \colon G → k$.
%\end{Def}
%
%Characters are constant on adjoint orbits, i.e.~are elements of $\O(G)^G$ (where the meaning of $\O$ depends on the context).

\begin{Def}
    An element of $\GL n$ is called \emph{semisimple} if it is diagonalizable.
    An element of $G$ is \emph{semisimple} if its image is so for any embedding of $G$ into a $\GL n$.
    The subset of all semisimple elements of $G$ is denoted by $G_s$.
\end{Def}

In general $G_s$ is neither closed nor a subgroup.

\begin{Def}
    An element of $\GL n$ is called \emph{unipotent} if all its eigenvalues are $1$.
    An element of $G$ is \emph{unipotent} if its image is so for any embedding of $G$ into a $\GL n$.
    The subset of all unipotent elements of $G$ is denoted by $G_u$.
    A group $G$ is called unipotent if $G = G_u$.
\end{Def}

The last two definitions are independent of the embedding of $G$ into $\GL n$ (and the choice of $n$).

\begin{Thm}[Jordan decomposition]
    Any element $g∈G$ can be written uniquely as a product $g=su$ with $s$ semisimple, $u$ unipotent and $[s,u] = 1$.
\end{Thm}

\begin{Def}
    The \emph{radical} $R(G)$ of $G$ is the maximal closed, connected, normal, solvable subgroup of $G$.
    If $R(G) = 1$, then $G$ is called \emph{semisimple}.
\end{Def}

Equivalently, $G$ is semisimple if it has no closed connected commutative normal subgroup except ${1}$.

\begin{Def}
    The \emph{unipotent radical} $R_u(G)$ of $G$ is the set of unipotent elements of $R(G)$.
    If $R_u(G) = 1$, then $G$ is called \emph{reductive}.
\end{Def}

$R_u(G)$ is a closed connected normal subgroup of $G$.

\begin{Ex}
    The group $\GL n$ is reductive and $R(\GL n)$ consists of the scalar matrices.
    The groups $\SL n$, $\SO n$ (for $n \ge 3$) and $\Sp{2m}$ are semisimple.
\end{Ex}

If $G$ is reductive, then $R(G) = Z(G)^\open$ is a torus \Humph{Lem.~19.5}.

\subsection*{tori}

\begin{Def}
    A \emph{torus} is an affine algebraic group which is isomorphic to a product of $\mathbb G_m = \GL 1$.
\end{Def}

\begin{Def}
    A \emph{character} of a torus $T$ is a morphisms of algebraic groups $T → \Gm$.
    The set of all characters if $X(T) = \Hom(T, \Gm)$.
    A \emph{cocharacter} or {one-parameter subgroup} of $T$ is a morphism of algebraic groups $\Gm → T$.
\end{Def}

If $T \cong \Gm^n$, then $X(T) = ℤ^n$.

\begin{Thm}[\Humph{Cor~16.3}]
    Let $T$ be a torus in $G$.
    Then $N_G(T)^\open = Z_G(T)^\open$, where $N_G(T)$ is the normalizer and $Z_G(T)$ the centralizer of $T$ in $G$.
\end{Thm}

\begin{Def}
    Let $T$ be a torus of $G$. 
    Then the \emph{Weyl group of $T$ in $G$} is
    \[
        W(T) = \rquot{N_G(T)}{Z_G(T)}.
    \]
    If $H$ is a maximal torus, then $W = W(H)$ is the \emph{Weyl group} of $G$.
\end{Def}

\begin{Thm}[\Humph{Cor.~21.3A}]
    Any two maximal tori of $G$ are conjugate.
\end{Thm}

Hence the Weyl group of $G$ is defined up to non-canonical isomorphism.

\begin{Prop}[\Humph{Prop.~24.1A}]
    The Weyl group of $G$ is finite.
\end{Prop}

\begin{Thm}[\Humph{Thm.~22.3}]
    If $H$ is a maximal torus, then $Z_G(H)$ is connected and hence equal to $H$.
\end{Thm}

If $T$ is a torus, then $W(T)$ has a natural action on $T$ by conjugation.

\begin{Def}
    The dimension of the maximal tori of $G$ is called the \emph{rank} of $G$.
\end{Def}

\begin{Prop}
    Let $φ\colon G → G'$ be an epimorphism of (connected) algebraic groups.
    Then $φ$ sends maximal tori to maximal tori and all maximal tori of $G'$ are obtained in this way.
\end{Prop}

\subsection*{solvable and nilpotent groups}

A group is called \emph{solvable} if its \emph{derived series} $D^{i}G = [D^{i-1}G,D^{i-1}G]$ terminates in $1$.
All $D^iG$ are closed normal subgroups of $G$. 
If $G$ is connected, then so are the $D^iG$.

\begin{Lem}
    Subgroups and homeomorphic images of solvable groups are solvable.
    If $N$ is a normal subgroup of $G$ with $N$ and $G/N$ solvable, then $G$ is solvable.
    If $A, B$ are normal solvable subgroups of $G$, then so is $AB$.
\end{Lem}

A group is called \emph{nilpotent} if its \emph{descending central series} $C^{i}G = [C^{i-1}G,G]$ terminates in $1$.
All $C^iG$ are closed normal subgroups of $G$. 
If $G$ is connected, then so are the $C^iG$.

\begin{Lem}[\Humph{Prop.~17.4}]
    Subgroups and homeomorphic images of solvable groups are solvable.
    If $\rquot{G}{Z(G)}$ is nilpotent, then so is $G$.
    If $G$ is nilpotent, then $\dim Z(G) \ge 1$ and if $H$ is a proper closed subgroup of $G$, then $\dim H < \dim N_G(H)$.
    In particular if $H$ is a closed subgroup of codimension $1$ in a nilpotent group, then $H$ is normal.
\end{Lem}

\begin{Thm}[\Humph{Cor.~17.5}]
    Every unipotent subgroup of of $\GL n$ is nilpotent.
\end{Thm}

If $G$ is connected and solvable, there is a split short exact sequence 
\[
    1 → G_u → G → T → 1,
\]
with $T$ a maximal torus \Humph{Thm.~19.3}. 
Thus $G \cong T ⋉ G_u$.
All maximal tori of $G$ are conjugate under $G^∞ = C^∞G$.

If $G$ is connected and nilpotent, then $G_s$ is a closed subgroup and $G \cong G_s × G_u$.

Moreover, if $H$ is a subgroup of a connected solvable $G$ with $H$ consisting of semisimple elements, then $H$ lies in a torus and $C_G(H) = N_G(H)$ is connected.

\subsection*{borel subgroups}

\begin{Def}
    A \emph{Borel subgroup} of $G$ is a maximal (automatically closed and connected) solvable subgroup of $G$.
\end{Def}

Note that the Borel subgroups of $G$ and $G^\conn$ coincide.
In the following statements it might be necessary to assume that $G$ is connected.

\begin{Thm}[\Humph{Thm.~21.3 and Thm.~23.1}]
    Let $B$ be a Borel subgroup.
    Then $\rquot{G}{B}$ is a projective variety and all Borel subgroups are conjugate to $B$.
    Further $N_G(B) = B$
\end{Thm}

The projective variety $\mathcal{Fl} = \rquot GB$ is called the \emph{flag variety of $G$}.
Its points are parametrize with the collection $\mathcal B$ of all Borel subgroups of $G$ (by sending each Borel to its unique fixed point in $\mathcal{Fl}$.
The action of $G$ on $\mathcal B$ by conjugation corresponds to the action of $G$ on $\mathcal Fl$ by left multiplication.
If $H$ is any subgroup of $G$, then the set $\mathcal B^H$ of Borel subgroups containing $H$ correspond to the set of fixed points of $H$ on $\mathcal{Fl}$.

The maximal tori (resp.~maximal connected unipotent subgroups) of $G$ are those of the Borel subgroups of $G$, and are all conjugate.

\begin{Prop}
    Let $φ\colon G → G'$ be an epimorphism of (connected) algebraic groups.
    Then $φ$ sends Borel subgroups (resp.~maximal connected unipotent subgroups) to subgroups of the same type and such subgroups of $G'$ are obtained in this way.
\end{Prop}

\begin{Thm}[\Humph{Thm~22.2}]
    Let $B$ be a Borel subgroup of $G$ and $H$ a maximal torus.
    Then the union of all conjugates of $B$ (resp.~$H$, resp.~$B_u$) is $G$ (resp.~$G_s$, resp.~$G_u$).
\end{Thm}

\begin{Prop}[\Humph{Cor.~22.2B}]
    Let $B$ be a Borel subgroup of $G$.
    Then $Z(B) = Z(G)$
\end{Prop}

\begin{Prop}[\Humph{Prop.~24.1A}]
    Let $H$ be a maximal torus with Weyl group $W$.
    Then $W$ permutes the set $\mathcal B^H$ simply transitively.
\end{Prop}

Setting $N = R_u(B) = B_u$ one gets a decomposition
\[
    B = HN, \qquad H∩N = \{1\}
\]
with $H$ a maximal torus.

\begin{Thm}[Bruhat decomposition]
    Let $B$ be a Borel subgroup of $G$ with corresponding maximal torus $H$ and Weyl group $W$.
    Then
    \[
        G = \coprod_{w ∈ W} BwB.
    \]
\end{Thm}

Note that the double coset $BwB$ does not depend on the choice of representative of $w$ in $N_G(H)$.

\begin{Def}
    The locally closed subvarieties $X_w = \rquot{BwB}B$ of $X = \mathcal{Fl}$ are called \emph{Schubert cells} and the closures $\overline{X_w}$ are called \emph{Schubert varieties} of $G$.
\end{Def}

\subsection*{parabolic subgroups}

\begin{Def}
    A subgroup $P$ of $G$ is called \emph{parabolic} if $\rquot GP$ is projective (or equivalently, complete).
\end{Def}

\begin{Prop}[\Humph{Cor.~21.3B}]
    A closed subgroup of $G$ is parabolic if and only if it contains a Borel subgroup of $G$.
    Hence a Borel subgroup of $G$ is a minimal parabolic subgroup.
\end{Prop}

\begin{Prop}
    The normalizer of a parabolic $P$ is equal to $P$. 
    In particular all parabolic subgroups are connected.
\end{Prop}

\begin{Prop}
    Let $P$ and $Q$ be conjugate parabolic subgroups both containing the Borel subgroup $B$.
    Then $P = Q$.

    In particular the set of conjugacy classes of parabolic subgroups is in bijection with parabolic subgroups containing a fixed Borel $B$.
\end{Prop}

\begin{Prop}
    Let $φ\colon G → G'$ be an epimorphism of (connected) algebraic groups.
    Then $φ$ sends parabolic subgroups to parabolic subgroups and all parabolic subgroups of $G'$ are obtained in this way.
\end{Prop}

\begin{Def}
    Let $P$ be a parabolic subgroup.
    Then $\rquot{P}{R_u(P)}$ is called a \emph{Levi} of $G$.
\end{Def}

Thus the set $[P]$ of all conjugates to $P$ in $G$ is isomorphic to $\rquot GP$.
Every Borel subgroups is conjugate

\begin{Def}
    Let $P$ be a parabolic subgroup of $G$.
    Then the projective variety $\rquot GP$ is called the \emph{partial flag variety} corresponding to $P$.
\end{Def}



\section*{examples}

\subsection*{\texorpdfstring{$G = \SL2$}{G = SL2}}

We have
\begin{align*}
    \SL2 &= \left\{\begin{psmallmatrix}a & b\\ c & d\end{psmallmatrix} : ad-bc = 1\right\}, \\
    B    &= \left\{\begin{psmallmatrix}a & b\\ 0 & d\end{psmallmatrix} : ad = 1\right\}, \\
    H    &= \left\{\begin{psmallmatrix}a & 0\\ 0 & d\end{psmallmatrix} : ad = 1\right\}, \\
    N    &= \left\{\begin{psmallmatrix}1 & b\\ 0 & 1\end{psmallmatrix}\right\}, \\
    W    &= \rquot{N_G(T)}{T} = 
                \left\{\begin{psmallmatrix}1 & 0\\ 0 & 1\end{psmallmatrix},
                \begin{psmallmatrix}0 & 1\\ -1 & 0\end{psmallmatrix}\right\} \cong \rquot ℤ2,\\
    \rquot GB &= \mathcal{Fl} = \left\{ \{0\} \subsetneq L \subsetneq ℂ²\right\} \cong \ps 1, \\
    W    &= \lquot{B}{\mathcal{Fl}} = \dquot BGB = \{ (1:0), \{(z:1)\} \}.
\end{align*}
The Bruhat decomposition is 
\[
G = B\begin{psmallmatrix}1 & 0\\ 0 & 1\end{psmallmatrix}B \amalg
B\begin{psmallmatrix}0 & 1\\ -1 & 0\end{psmallmatrix}B,
\] 
with
\[
    B\begin{psmallmatrix}1 & 0\\ 0 & 1\end{psmallmatrix}B = B
    \qquad\text{and}\qquad
    B\begin{psmallmatrix}0 & 1\\ -1 & 0\end{psmallmatrix}B = \left\{\begin{psmallmatrix} a & b \\ c & d\end{psmallmatrix} : ad-bc = 1 \text{ and } c \ne 0\right\}.
\]

\printbibliography
\end{document}
