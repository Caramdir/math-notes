%type: notes
%title: Hodge theory
%tags: Hodge structures, polarized Hodge structures
\documentclass[english]{short-notes}

\addbibresource{math.bib}
\usepackage{math-general}
\usepackage{math-ag}

\newcommand\A{\sheaf A}

\title{Hodge Theory}
\author{Clemens Koppensteiner}
\date{April 11, 2011}


\begin{document}

\maketitle

\tableofcontents

\section{Hodge Structures}
    
\subsection{Definition}

\begin{Def}
    A \emph{Hodge structure $V$ of weight $k$} consists of the following data:
    \begin{itemize}
        \item A free Abelian group $V_ℤ$ of finite rank.
        \item A decomposition of complex vector spaces $V_ℂ = V_ℤ ⊗_ℤ ℂ = \bigoplus_{p+q=k} V^{p,q}$ with $\overline{V^{p,q}} = V^{q,p}$.
            The indices $p,q$ can run over all integers.
    \end{itemize}
    The decomposition induces the \emph{Hodge filtration} $F$ of $V_ℂ$ by $F^pV_ℂ = \bigoplus_{r≥p} V^{r,k-r}$.
\end{Def}

One can recover $V^{p,q}$ from the filtration by $V^{p,q} = F^pV_ℂ ∩ \overline{F^q V_ℂ}$.

\begin{Def}
    A morphism of Hodge structures $V → W$ is a morphism of Abelian groups $φ\colon V_ℤ → W_ℤ$ such that $φ_ℂ\colon V_ℂ → W_ℂ$ is compatible with the Hodge filtration, i.e.\ $φ_ℂ(F^pV_ℂ) ⊆ F^pW_ℂ$.
\end{Def}

\begin{Ex}
    Let $V$ be a Hodge structure of weight $-1$ with $F^pV_ℂ = V_ℂ$ for $p ≥ 1$, i.e.\ $V_ℂ = V^{-1,0} \oplus V^{0,-1}$.
    As $\dim_ℂ V^{-1,0} = \dim_ℂ V^{0,-1}$, the free Abelian group $V_ℤ$ has rank $2n$ for some integer $n$.
    The composition
    \[ 
    V_ℤ → V_ℂ \onto V_{0,-1}
    \]
    is injective, as the image is fixed by conjugation.
    Thus $V_ℤ$ forms a lattice in $V^{0,-1}$, so that $\rquot{V^{0,-1}}{V_ℤ}$ is a complex torus of dimension $1$.

    Conversely, let $X = W/Λ$ be a complex torus of dimension $n$.
    Setting $V^{0,-1} = W$ in $Λ_ℂ$ gives a Hodge structure of weight $-1$ on $V_ℤ = Λ$.
    Clearly the constructions are inverse and functorial.
    Note that $Λ$ can be identified with $H₁(X,ℤ)$ and $W$ with the tangent space $T₀X \cong H¹(X,\O_X)^*$.

    We could also look at pure Hodge structures of weight $k$ with Hodge filtration vanishing for negative indices.
    This gives a complex torus in exactly the same way.
    But now the natural identifications are $V = H¹(X,ℤ)$, $V^{1,0} = H^0(X,Ω_X¹)$ and $V^{0,1} = H¹(X,\O_X)$, which give an exact sequence
    \[ 0 → H^0(X,Ω_X¹) → H¹(X,ℂ) → H¹(X,\O_X) → 0. \]
    The quotient $\rquot{H¹(X,\O_X)}{H¹(X,ℤ)}$ is not the torus $X$, but the dual torus $\hat X$.
    Dualizing, we again obtain an isomorphism of categories.
\end{Ex}

The tensor product of two Hodge structures is given by the tensor product on the integral structure with the Hodge filtration given by the tensor product of filtrations.
Concretely, if $V$ is a Hodge structure of weight $k$ and $W$ one of weight $l$, then $V⊗W$ is of weight $k+l$ with
\[
\left( V ⊗ W \right)^{p,q} = \bigoplus_{\mathclap{\substack{p₁+p₂ = p\\q₁+q₂=q}}} V^{p₁,q₁} ⊗ V^{p₂,q₂}.
\]

\begin{Def}
    The \emph{Tate Hodge structure} $ℤ(1)$ is the Hodge structure of weight $-2$ and rank $1$, purely in bidegree $(-1,-1)$ given by $2π\i ℤ ⊆ ℂ$.
    The tensor power $ℤ(1)^{⊗n}$ ($n ∈ ℤ$) is denoted $ℤ(n)$. 
    It is of weight $-2n$, rank $1$, purely in bidegree $(-n,-n)$, given by $(2π\i)^n ℤ ⊆ ℂ$.
\end{Def}

\subsection{Hodge Theory of Kähler Manifolds}

Let $X$ be a Kähler manifold of dimension $n$.
Let $V_ℤ = H^k(X,ℤ)$.
Then $V_ℂ = V_ℤ ⊗_ℤ ℂ = H^k(X, ℂ)$.
By Hodge theory, there exists a decomposition $H^k(X,ℂ) = \bigoplus_{p+q=k} H^{p,q}$.
If we identify elements of $H^k(X,ℂ)$ with classes of differential forms via de Rham cohomology, then $H^{p,q}$ are exactly the classes of $(p,q)$-forms.
Further, one can identify $H^{p,q}$ with $H^q(X,Ω^p)$.

One way to look at this is via the \emph{Hodge-de Rham spectral sequence} (or \emph{Fröhlicher spectral sequence}).
Let $(Ω_X^\cx,∂)$ be the holomorphic de Rham complex and give it the \enquote{naive} filtration $F^pΩ_X^\cx = Ω_X^{≥p}$.
The holomorphic de Rham complex is quasi-isomorphic to the de Rham complex $(\sheaf A_X^\cx,\mathrm d)$ as a resolution of $ℂ$.
Thus the cohomology groups of its global sections are $H^k(X,ℂ)$.
To calculate these groups, we can pick a acyclic resolution of the holomorphic de Rham complex into a double complex, apply the global sections functor and compute the spectral sequence.
Of course the natural resolution is by Dolbeault complexes:
\[
\begin{tikzpicture}
    \matrix (m) [commutative diagram] {
          & \vdots     & \vdots     & \vdots     &  \\
        0 & \A_X^{0,2} & \A_X^{1,2} & \A_X^{2,2} & \cdots \\
        0 & \A_X^{0,1} & \A_X^{1,1} & \A_X^{2,1} & \cdots \\
        0 & \A_X^{0,0} & \A_X^{1,0} & \A_X^{2,0} & \cdots \\
        ℂ & \O_X       & Ω_X¹       & Ω_X²       & \cdots \\
        };
    \draw[->]
        (m-2-1) edge node[above] {$∂$} (m-2-2)
        (m-2-2) edge node[above] {$∂$} (m-2-3)
        (m-2-3) edge node[above] {$∂$} (m-2-4)
        (m-2-4) edge node[above] {$∂$} (m-2-5)
        (m-3-1) edge node[above] {$∂$} (m-3-2)
        (m-3-2) edge node[above] {$∂$} (m-3-3)
        (m-3-3) edge node[above] {$∂$} (m-3-4)
        (m-3-4) edge node[above] {$∂$} (m-3-5)
        (m-4-1) edge node[above] {$∂$} (m-4-2)
        (m-4-2) edge node[above] {$∂$} (m-4-3)
        (m-4-3) edge node[above] {$∂$} (m-4-4)
        (m-4-4) edge node[above] {$∂$} (m-4-5)
        (m-5-1) edge node[above] {$∂$} (m-5-2)
        (m-5-2) edge node[above] {$∂$} (m-5-3)
        (m-5-3) edge node[above] {$∂$} (m-5-4)
        (m-5-4) edge node[above] {$∂$} (m-5-5)
        (m-5-2) edge node[right] {$\cconj ∂$} (m-4-2)
        (m-4-2) edge node[right] {$\cconj ∂$} (m-3-2)
        (m-3-2) edge node[right] {$\cconj ∂$} (m-2-2)
        (m-2-2) edge node[right] {$\cconj ∂$} (m-1-2)
        (m-5-3) edge node[right] {$\cconj ∂$} (m-4-3)
        (m-4-3) edge node[right] {$\cconj ∂$} (m-3-3)
        (m-3-3) edge node[right] {$\cconj ∂$} (m-2-3)
        (m-2-3) edge node[right] {$\cconj ∂$} (m-1-3)
        (m-5-4) edge node[right] {$\cconj ∂$} (m-4-4)
        (m-4-4) edge node[right] {$\cconj ∂$} (m-3-4)
        (m-3-4) edge node[right] {$\cconj ∂$} (m-2-4)
        (m-2-4) edge node[right] {$\cconj ∂$} (m-1-4);
\end{tikzpicture}
\]
We take global sections.
The zeroth page of the spectral sequence, $(E_0^{p,q},d_0)$ is 
\[
\begin{tikzpicture}
    \matrix (m) [commutative diagram] {
        \vdots  & \vdots     & \vdots     & \vdots     &  \\
        0 & \A_X^{0,2}(X) & \A_X^{1,2}(X) & \A_X^{2,2}(X) & \cdots \\
        0 & \A_X^{0,1}(X) & \A_X^{1,1}(X) & \A_X^{2,1}(X) & \cdots \\
        0 & \A_X^{0,0}(X) & \A_X^{1,0}(X) & \A_X^{2,0}    & \cdots \\
        0 & 0             & 0             &  0            & \cdots \\
        };
    \draw[->]
        (m-5-1) edge (m-4-1)
        (m-4-1) edge (m-3-1)
        (m-3-1) edge (m-2-1)
        (m-2-1) edge (m-1-1)
        (m-5-2) edge (m-4-2)
        (m-4-2) edge node[right] {$\cconj ∂$} (m-3-2)
        (m-3-2) edge node[right] {$\cconj ∂$} (m-2-2)
        (m-2-2) edge node[right] {$\cconj ∂$} (m-1-2)
        (m-5-3) edge (m-4-3)
        (m-4-3) edge node[right] {$\cconj ∂$} (m-3-3)
        (m-3-3) edge node[right] {$\cconj ∂$} (m-2-3)
        (m-2-3) edge node[right] {$\cconj ∂$} (m-1-3)
        (m-5-4) edge (m-4-4)
        (m-4-4) edge node[right] {$\cconj ∂$} (m-3-4)
        (m-3-4) edge node[right] {$\cconj ∂$} (m-2-4)
        (m-2-4) edge node[right] {$\cconj ∂$} (m-1-4);
\end{tikzpicture}
\]
The next page, $(E₁^{p,q},d₁)$ is
\[
\begin{tikzpicture}
    \matrix (m) [commutative diagram] {
        \vdots & \vdots      & \vdots      & \vdots       &  \\
        0      & H^2(X,\O_X) & H^2(X,Ω_X¹) & H^2(X,Ω_X²) & \cdots \\
        0      & H^1(X,\O_X) & H^1(X,Ω_X¹) & H^1(X,Ω_X²) & \cdots \\
        0      & H^0(X,\O_X) & H^0(X,Ω_X¹) & H^0(X,Ω_X²) & \cdots \\
        0      & 0           & 0           & 0            & \cdots \\
        };
    \draw[->]
        (m-2-1) edge (m-2-2)
        (m-2-2) edge node[above] {$d₁$} (m-2-3)
        (m-2-3) edge node[above] {$d₁$} (m-2-4)
        (m-2-4) edge node[above] {$d₁$} (m-2-5)
        (m-3-1) edge (m-3-2)
        (m-3-2) edge node[above] {$d₁$} (m-3-3)
        (m-3-3) edge node[above] {$d₁$} (m-3-4)
        (m-3-4) edge node[above] {$d₁$} (m-3-5)
        (m-4-1) edge (m-4-2)
        (m-4-2) edge node[above] {$d₁$} (m-4-3)
        (m-4-3) edge node[above] {$d₁$} (m-4-4)
        (m-4-4) edge node[above] {$d₁$} (m-4-5)
        (m-5-1) edge (m-5-2)
        (m-5-2) edge (m-5-3)
        (m-5-3) edge (m-5-4)
        (m-5-4) edge (m-5-5);
\end{tikzpicture}
\]
Now any further step can only decrease the dimensions of the spaces in the sequence.
We know that $\bigoplus_{p+q=k} E^{p,q}_∞$ must be equal to $H^k(X,ℂ)$. 
Hodge theory tells us this is already true for the first page, so that the spectral sequence degenerates here and $d₁ = 0$.
Note that the \enquote{naive} filtration on the holomorphic de Rham complex induces the Hodge filtration via this spectral sequence.

The upshot of looking at Hodge theory this way, is that the process is entirely algebraic and can be carried over to varieties that are not defined over $ℂ$ (the spectral sequence might not degenerate though).

\subsection{Hodge Structures Via Actions of \texorpdfstring{$ℂ^*$}{ℂ*}}

Let $S$ be the \emph{real} algebraic group $ℂ^*$, i.e.\ $S$ is obtained from $\mathbb{G}_m$ by Weil restriction of scalars from $ℂ$ to $ℝ$.
Then a Hodge structure $V$ induces an algebraic action of $S$ on $V_ℝ$ by setting letting $z$ act on $V^{p,q}$ as $z^p\cconj z^q$.
Conversely, if we have a real algebraic action of $S$ on a real vector space $V$, we obtain a decomposition of $V_ℂ = V ⊗_ℝ ℂ$ into $V^{p,q}$ by requiring that $z$ acts on $V^{p,q}$ by multiplication with $z^p\cconj z^q$.
Then $V^{p,q} = \cconj{V^{q,p}}$.

\begin{Def}
    A \emph{real Hodge structure} consists of a finite dimensional real vector space $V$ with an action of the real algebraic group $S$ on $V$.
    It is \emph{of weight $k$}, if $V^{p,q} = 0$ for $p+q \ne k$.
\end{Def}

This gives an alternative, equivalent, definition of a Hodge structure:

\begin{Def}
    A \emph{Hodge structure} of weight $k$ consists of a free Abelian group $V_ℤ$ of finite rank and real Hodge structure of weight $k$ on $V_ℝ = V ⊗_ℤ ℝ$.
\end{Def}

\section{Polarized Hodge Structures}

\subsection{Definition}

Let $V$ be a Hodge structure. 
Then the action of $\i ∈ S$ on $V_ℝ$ induces an automorphism $C$.
Concretely, $C$ acts on $V^{p,q}$ by multiplication with $\i^{p-q}$.

\begin{Def}
    A \emph{polarization} of a Hodge structure $V$ of weight $k$ is given by a morphism
    \[
    (\cdot,\cdot)\colon V ⊗ V → ℤ(-k),
    \]
    such that the real bilinear form $(2π\i)^k(x,Cy)$ is symmetric and positive definite on the real part of $V^{p,q} \oplus V^{q,p}$.
\end{Def}

Let us unravel this definition.
The Hodge structure $V ⊗ V$ is of weight $2k$.
As $Z(-k)$ is purely of bidegree $(k,k)$, a morphism is given by a map $(V⊗V)^{k,k} → ℤ(-k)^{k,k}$, respecting the integral structure.
Now,
\[ 
\left( V ⊗ V \right)^{k,k} =
\bigoplus_{\mathclap{p+q=k}} V^{p,q} ⊗ V^{q,p} =
\bigoplus_{\mathclap{p+q=k}} V^{p,q} ⊗ \cconj{V^{p,q}}.
\]
Thus if $Q(x,y)$ denotes the homomorphism $V_ℤ ⊗ V_ℤ → ℤ$ underlying the morphism above, then the Hodge structure is orthogonal with respect to $Q_ℂ(x,\cconj y)$.
Further, as $(x,Cy) = (2π\i)^kQ_ℝ(x,Cy)$ is symmetric, we have $(x,y) = (Cx,Cy) = (y,C²x) = (-1)^k(y,x)$ so that $Q$ is symmetric if $k$ is even and alternating otherwise.
Let $x ∈ V^{p,q}$, then the positive definiteness requirement, orthogonality and symmetry imply that
\begin{multline*}
    0 < \i^k\i^k Q_ℝ(x+\bar x, C(x + \bar x)) =
    (-1)^k\left(Q_ℝ(x,C\bar x) + Q_ℝ(\bar x, Cx)\right) = \\ =
    2(-1)^k Q_ℝ(x,C\bar x) = 
    2(-1)^k\i^{q-p}ℚ_ℝ(x,\bar x) =
    2\i^{p-q}ℚ_ℝ(x,\bar x).
\end{multline*}
Thus we can alternatively define a polarization in the following way.
\begin{Def}
    A polarization on a Hodge structure $V$ is a bilinear form $Q$ on $V_ℤ$, such that, when extended linearly,
    \begin{itemize}
        \item $Q(x,y) = (-1)^kQ(y,x)$;
        \item the Hodge decomposition is orthogonal with respect to $Q$;
        \item $\i^{p-q}Q(x,\bar x) > 0$ for $x ∈ V^{p,q}$.
    \end{itemize}
\end{Def}

\subsection{Lefschetz Decomposition And Intersection Form}

Let $X$ be a compact Kähler manifold of dimension $n$ with Kähler form $ω$.
Then cupping with the class $[ω]$ gives a morphism $L\colon H^k(X,ℂ) → H^{k+2}(X,ℂ)$.
Since $ω$ is of bidegree $(1,1)$, this morphism is compatible with the Hodge structure on $H^*(X,ℂ)$.

Write $Λ$ for the formal adjoint of $L$.
Then it is known that $[L,Λ]$ acts on $H^k(X,ℂ)$ by multiplication with $k-n$.
Set $V^k = H^{k+n}(X,ℂ)$ for $k=-n,\dotsc,n$ and set $V = \bigoplus V^k$.
Then the operators $L$ and $Λ$ act on $V$. 
Further let $M$ be the operator on $V$ that is multiplication by $k$ on $V^k$.
\[
\begin{tikzpicture}
    \matrix[commutative diagram] (m) {
    V^{-k} & V^{-k+1} & \cdots\!\cdots & V^{k-1} & V^{k} \\
    };
    \draw[->,bend left=45]
        (m-1-1.north east) edge node[above] {$L$} (m-1-2.north west)
        (m-1-2.north east) edge node[above] {$L$} (m-1-3.north west)
        (m-1-3.north east) edge node[above] {$L$} (m-1-4.north west)
        (m-1-4.north east) edge node[above] {$L$} (m-1-5.north west)
        (m-1-5.south west) edge node[below] {$Λ$} (m-1-4.south east)
        (m-1-4.south west) edge node[below] {$Λ$} (m-1-3.south east)
        (m-1-3.south west) edge node[below] {$Λ$} (m-1-2.south east)
        (m-1-2.south west) edge node[below] {$Λ$} (m-1-1.south east);
    \draw[->] 
        (m-1-1) edge[loop above] node[above] {$M$} (m-1-1)
        (m-1-2) edge[loop above] node[above] {$M$} (m-1-3)
        (m-1-4) edge[loop above] node[above] {$M$} (m-1-4)
        (m-1-5) edge[loop above] node[above] {$M$} (m-1-5);
\end{tikzpicture}
\]
One immediately verifies the following relations:
\[
[L,Λ] = M, \qquad
[L,M] = 2L, \qquad
[Λ,M] = -2L.
\]
Thus $L$, $Λ$ and $M$ form an $\mathfrak{sl}_2$-triple, i.e.\ the $ℂ$-algebra generated by them is the complexified Lie algebra of $\SL2ℝ$.
So we have a representation of the complexified Lie algebra of $\SL2ℝ$ on $V$.
By Lie theory, this representation splits into irreducible representations.
The Lie algebra $\mathfrak{sl}_2$ has exactly on irreducible representation of dimension $j$ for each positive integer $j$.
The representation of weight $j$ decomposes into $j$ eigenspaces for $M$ with eigenvalues $-j,-j+2,\dotsc,j$.
The operator $L$ maps the eigenspace for $k$ to the eigenspace for $k+2$ and $Λ$ goes the other way.
In particular, on the representation of weight $n$, $L^{j+1} = 0$, but $L^{j} \ne 0$.

This can be used to find the decomposition of $V$ into irreducible representations, by finding the vectors (called lowest weight vectors) for which $L^{j+1}=0$, but $L^j \ne 0$.
Note that we have already graded $V$ by the eigenvalues of $M$, so that $L^j \ne 0$ on $V^k$ if and only if $k \le -j$.
In terms of cohomology this idea translates to the following definition.
\begin{Def}
    A cohomology class $α ∈ H^k(X,ℂ)$, $k ≤ n$ is called \emph{primitive} if $L^{n-k+1}α=0 ∈ H^{2n-k+1}$.
    Equivalently, $α$ is primitive if $Λα = 0$.
\end{Def}

Thus the decomposition of $V$ into irreducible representations of $\mathfrak{sl₂}$ immediately implies the \emph{Lefschetz} decomposition:
Every cohomology class $α ∈ H^k(X,ℂ)$ admits a unique decomposition $α = \sum_r L^r α_r$ with $α_r$ a primitive element of $H^{k-2r}(X,ℂ)$ ($k-2r ≤ \min\{n,2n-k\}$).

As $ω$ is a real form, the same is true for $H^k(X,ℝ)$.
If $ω$ is integral, i.e.\ belongs to $H²(X,ℤ) ⊆ H²(X,ℝ)$, then the same is true for $H^k(X,ℤ)$.
Further, since $ω$ is a $(1,1)$-form, the decomposition is compatible with the Hodge decomposition.

Now assume that $ω$ is integral.
Let $H^k(X,ℂ)_{\mathrm{prim}}$ be the primitive elements of $H^k(X,ℂ)$ and use similar notations for cohomology with coefficients in $ℤ$.
Then the $H^k(X,ℤ)_{\mathrm{prim}} ⊆ H^k(X,ℂ)_{\mathrm{prim}}$ is a Hodge structure of weight $k$ with decomposition $H^{p,q}_{\mathrm{prim}} = H^{p,q} ∩ H^k(X,ℂ)_{\mathrm{prim}}$.
Define bilinear form $Q$ on $H^k(X,ℂ)$ by 
\[
Q(α,β) = (-1)^{\frac{k(k-1)}2}∫_X L^{n-k}α ∧ β = (-1)^{\frac{k(k-1)}2}∫_X ω^{n-k} ∧ α ∧ β.
\]
As $ω$ is integral, $Q$ takes integral values on integral classes.
One shows that $Q$ is a polarization of $H^k_{\mathrm{prim}}$.
\end{document}
