\documentclass{ck-article}

\usepackage{math-ag, math-alg}

\newcommand\loopliealg[1]{\mathcal{L}\liealg{#1}}
\newcommand\affliealg[1]{\hat{\liealg{#1}}}
\newcommand\affKMliealg[1]{\widehat{\loopliealg{#1}}}
\DeclareMathOperator\Res{Res}
\DeclareMathOperator\Ind{Ind}
\newcommand\SymGrp[1]{S_{#1}}
\newcommand\triv{{\mathrm{triv}}}
\DeclareMathOperator\Heis{Heis}
\newcommand{\normord}[1]{\mathopen{:}#1\mathclose{:}}

% symbols not available in font:
\renewcommand\lParen{((}
\renewcommand\rParen{))}


\title{Heisenberg algebras}
\author{Clemens Koppensteiner}

\addbibresource{math.bib}

\begin{document}

\maketitle

\section{Heisenberg algebras}

Recall that the \emph{Heisenberg group} is the subgroup of $\SL 3$ consisting of upper triangular matrices of the form
\[
    \begin{pmatrix}
        1 & a & b \\
        0 & 1 & c \\
        0 & 0 & 1
    \end{pmatrix}.
\]
The \emph{Heisenberg Lie algebra} $\mathfrak{h}$ is the corresponding Lie algebra, consisting of matrices of the form
\[
    \begin{pmatrix}
        0 & a & b \\
        0 & 0 & c \\
        0 & 0 & 0
    \end{pmatrix}.
\]
Thus it has a basis given by
\[
    a_1 = 
    \begin{pmatrix}
        0 & 1 & 0 \\
        0 & 0 & 0 \\
        0 & 0 & 0
    \end{pmatrix}, \quad
    a_{-1} = 
    \begin{pmatrix}
        0 & 0 & 0 \\
        0 & 0 & 1 \\
        0 & 0 & 0
    \end{pmatrix}, \quad
    c = 
    \begin{pmatrix}
        0 & 0 & 1 \\
        0 & 0 & 0 \\
        0 & 0 & 0
    \end{pmatrix}
\]
with relations
\[
    [c,\, a_1] = [c,\, a_{-1}] = 0 \quad\text{and}\quad [a_1,\, a_{-1}] = c.
\]
The \emph{Heisenberg algebra} is the enveloping algebra $U(\mathfrak{h})$ of the Heisenberg Lie algebra.
Thus it is the associative algebra with generators $c$, $a_{-1}$ and $a_1$, and the above relations.
The elements $c$ (sometimes denoted $z$ or $\hbar$) is central.
From now on we often set $c = 1$.

\begin{Remark}
    The Heisenberg algebra modulo $c = 1$ is exactly the (1-dimensional) Weyl algebra (which is usually written with generators $x$ and $\partial_x$).
\end{Remark}

\subsection{The infinite dimensional Heisenberg algebra}\label{sec:infinite_Heisenberg}

Generalizing the above definition, one obtains higher dimensional analogues, and eventually the \emph{infinite Heisenberg Lie algebra} which is generated by $a_n$ for $n \in \ZZ \setminus \{0\}$ and central element $c$ with relations
\[
    [a_m,\, a_n] = \delta_{m,-n}c.
\]
The \emph{infinite Heisenberg algebra} is the enveloping algebra of this Lie algebra.

There are several variants of this construction:
\begin{itemize}
    \item The relations are often scaled as $[a_m,\, a_n] = m\delta_{m,-n}c$, or $[a_m,\, a_n] = n\delta_{m,-n}c$.
    % [CL,2.2.1] Have $n\delta_{m,-n}$ instead. So there might be a sign in the corresponding expressions for p and q below.
    \item A common labeling of the generators is $p_n = a_n$, $q_n = a_{-n}$ for $n \ge 1$, but we will use these letters for a different set of generators below.
    \item \cite[Definition~4.1]{Gordon:2009:InfiniteDimensionalLieAlgebras} adds an additional generator $a_0$, satisfying the same relations. Thus in this case the Heisenberg Lie algebra is a central extension of the loop Lie algebra of a 1-dimensional abelian Lie algebra.
    \item We often identify the \emph{central charge} $c$ with $1$.
\end{itemize}

% polynomials, symmetric groups

\subsection{Heisenberg algebras associated to a bilinear form}

Given a vector space $V$ with a symmetric bilinear form $\chi\colon V \times V \to V$.
Consider $L_V = V^{\ZZ \setminus \{0\}}$ and denote by $a_{v}(n)$ the image of $v \in V$ in the $n$th summand of $L_V$.
The (infinite) Heisenberg algebra $H_V$ of $(V, \chi)$ is the quotient of the tensor algebra $T(L_V) = \bigoplus_{i \ge 0} L_V^{\otimes i}$ by the relations
\[
    [a_v(m),\, a_w(n)] = m\chi(v,w)\delta_{m,-n}.
\]
\begin{Remarks}
    Again, there are renormalized versions where the left hand side of the relations do not contain the factor $m$, or differ by a sign.
    Similarly, one can again add a central element $c$.

    If $\chi$ is degenerate, then $H_V$ has a non-trivial center.
    Thus one often assumes $\chi$ to be non-degenerate.
\end{Remarks}

\begin{Example}
    If $V = k$ with inner product given by multiplication, one recovers the infinite Heisenberg algebra of Section~\ref{sec:infinite_Heisenberg}.
\end{Example}

% finite-dimensional Heisenberg Lie-algebra.

\subsection{Heisenberg algebras associated to a Dynkin diagram}\label{sec:heisenberg:Dynkin}

Let $\Gamma = (I, E)$ be a simply laced Dynkin diagram with vertices $I$ and edges $E$.
Consider the vector space $V = k^I$.
The Cartan matrix of $\Gamma$ defines a symmetric linear form on $V$:
\[
    \chi(i,j) =
    \begin{cases}
        \phantom{-}2 & \text{if } i = j, \\
        -1 & \text{if } i \ne j \text{ are connected by an edge}, \\
        \phantom{-}0 & \text{if } i \ne j \text{ are not connected by an edge.} \\
    \end{cases}
\]
The Heisenberg algebra of $\Gamma$ is $H_\Gamma = H_V$.

% More general situation?

\subsection{The Fock space}

Let $(V,\psi)$ be a vector space endowed with a symmetric bilinear form and let $H = H_V$ be the associated Heisenberg algebra with generators $a_v(n)$ with relation
\[
    [a_v(m),\, a_w(n)] = m\psi(v,w) \delta_{m,-n}.
\]
The \emph{Fock space representation} of $H$ is the quotient $F = H/I$, where $I$ is the left ideal generated by all $a_v(n)$ for $n > 0$.

Alternatively, let $H^+$ be the subalgebra generated by $a_v(n)$, $n \ge 0$, and let $\triv_0$ be the trivial representation of $H^+$ where all $a_i(n)$ act by $0$ (and if it exists, $a_v(0)$ acts as the identity[?]).
Then $F = \Ind_{H^+}^H(\triv_0)$.

Explicitly for $n > 0$, $a_v(-n)$ acts on $F$ by multiplication (called a \emph{creation operator}).
One computes that the \emph{annihilation operator} $a_v(n)$ acts by $a_v(n) \cdot a_w(-m) = m\delta_{n,m}\psi(v,w)1$.
The element $1 \in H$ is called the \emph{vacuum vector} and $H \cdot 1 = F$.

\begin{Example}
    If $\dim V = 1$, one can identify $F$ with the polynomial ring $k[x_1,x_2,\dots]$ in infinitely many variables via $a(-n) \mapsto x_n$.
    Thus $a(-n)$ acts by multiplication with $x_n$ and $a(n) \cdot x_i = n\delta_{n,i}$. [write down action on $x_i^k$?]
\end{Example}


\subsection{A different set of generators}

\subsection{Idempotent modification}

\section{The Frenkel--Kac construction}

The Frenkel--Kac constructions constructs the basic representation of an affine Lie algebra from the Fock space representation of a Heisenberg algebra.

\subsection{Affine Lie algebras}

Let $\liealg g$ be a complex simple finite dimensional Lie algebra with a choice of Cartan $\liealg h$ and positive roots $\Delta_+ \subset \Delta$.
Let $\kappa$ be the Killing form of $\liealg g$.

The vector space $\loopliealg g = \CC[t, t^-1] \otimes_\CC \liealg g$ has a Lie bracket defined by 
\[
    \bigl[ p \otimes X,\, q \otimes Y\bigr] = pq \otimes [X,\,Y].
\]
It is called the \emph{loop algebra} of $\liealg g$.
The Killing form $\kappa$ extends to a $\CC[t,t^{-1}]$-valued form on $\loopliealg g$.

The loop algebra has a unique central extension \cite[Theorem~10.2]{Gordon:2009:InfiniteDimensionalLieAlgebras}.
It is given by the \emph{affine Lie algebra} $\affliealg g = \loopliealg g \oplus \CC c$ with bracket
\[
    \bigl[p \otimes X + \alpha c,\, q \otimes Y + \beta c\bigr]  =
    [p\otimes X,\, q \otimes Y] + \Res\biggl(\frac{dp}{dt}q\biggr)\kappa(X,Y)c,
\]
where $\Res$ denotes the residue of a Laurent polynomial (i.e.~the coefficient of $t^{-1}$).
More explicitly,
\[
    \bigl[p \otimes t^m + \alpha c,\, q \otimes t^n + \beta c\bigr]  =
    t^{m+n} \otimes [X,\, Y] + m\delta_{m,-n}\kappa(X,Y)c.
\]
The Lie algebra $\affliealg g$ admits a derivation $d$ which acts by
\[
    d\bigl(p \otimes X + \alpha c) = t\frac{dp}{dt} \otimes X.
\]
One sometimes enlarges $\affliealg g$ by a semidirect product with $\CC d$ such that
\[
    [d, A] = d(A).
\]
The resulting Lie algebra is called the \emph{affine Kac--Moody Lie algebra} $\affKMliealg g$.

\subsection{The homogeneous Heisenberg subalgebra}

Either of the two subalgebras of $\affliealg g$
\begin{align*}
    \hat{\liealg s} & = \sum_{k \in \ZZ \setminus 0} (t^k \otimes_\CC \liealg h) \oplus \CC c\\
    \intertext{and}
    \tilde{\liealg s} & = \CC[t,t^{-1}] \otimes_\CC \liealg h \oplus \CC c
\end{align*}
are called a \emph{homogeneous Heisenberg subalgebra}.

We note that up to rescaling and a choice of positive roots, the Cartan matrix of the Dynkin diagram $\Gamma$ of $\liealg g$ induces the same symmetric bilinear form on $\liealg h$ as the Killing form.
Thus $\hat{\liealg s}$ is isomorphic to the Heisenberg algebra $H_\Gamma$ of Section~\ref{sec:heisenberg:Dynkin}.

\subsection{Vertex operators}

Let $F$ be the Fock space of $\hat{\liealg s}$ and $Y$ be the root lattice of $\liealg g$.
We set
\[
    V = F \otimes_\CC \CC_\epsilon(Y) \cong \bigoplus_{\lambda \in Y} F.
\]
Here $\CC_\epsilon(Y)$ is the group algebra of $Y$, with multiplication twisted by a sign [Why?].
Write $e_\lambda$ for the generators of $\CC(Y)$.

The goal is to give $V$ the structure of a representation of $\affliealg g$.
To do so we introduce \emph{vertex operators}.
Specifically, we consider the formal sum
\begin{equation}\label{eq:vertex_operators}
    X(i, z) = e_{\alpha_i} z^{\alpha_i}
    \exp\biggl(\sum_{n=1}^\infty \frac{z^n}{n} a_i(-n)\biggr)
    \exp\biggl(-\sum_{n=1}^\infty \frac{z^{-n}}{n} a_i(n)\biggr).
\end{equation}
Here $a_i(n) \in \hat{\liealg s}$ are the generators arising from the presentation as $H_\Gamma$ and $\alpha_i$ are the simple roots corresponding to the vertices of the Dynkin diagram $\Gamma$.
The operator $z^{\alpha_i}$ takes any element of the space $F \otimes e_\lambda$ to the monomial $z^{\langle \alpha_i, \lambda\rangle}$.

\begin{Remark}
    Formula \eqref{eq:vertex_operators} is taken from Wikipedia\footnote{\url{https://en.wikipedia.org/wiki/Vertex_operator_algebra\#Vertex_operator_algebra_defined_by_an_even_lattice}}.
    It seems to me that compared to \cite[Equation~(2.23)]{FrenkelKac:1980:BasicRepresentationsOfAffineLieAlgebras} it is off by a factor of $z$.
\end{Remark}

Write
\[
    X(i, z) = \sum_{k \in \ZZ} X_k(i) z^k.
\]
Defining the action of $E_{i,m} \in \affliealg g$ on $V$ to be $X_m(i)$ gives $V$ the structure of a $\affliealg g$-representation.
As such it is isomorphic to the basic representation $V_{\Lambda_0}$.
For details see \cite[Theorem~1]{FrenkelKac:1980:BasicRepresentationsOfAffineLieAlgebras}.

\begin{Remark}
    Starting from an arbitrary even lattice one can apply the same construction with the associated Heisenberg algebra and Fock space and still obtain vertex operators.
    Of course in general they will not give a representation of an affine Lie algebra.
\end{Remark}

\subsection{Virasoro operators}

Fix an orthonormal basis $h_1,\dotsc, h_n$ of $\liealg h$ and let $h_i(m)$ be the corresponding generators for $\tilde{\liealg s}$.
Define operators
\begin{align*}
    D_0 &= -\sum_{k \ge 1} \sum_{i=1}^n h_i(-k)h_i(k) - \frac12\sum_{i=1}^n h_i(0)^2 \\
    \intertext{and}
    D_m &= -\frac12 \sum_{k \in \ZZ} \sum_{i=1}^n h_i(-k)h_i(k+m)& \text{for } m \in \ZZ \setminus 0
\end{align*}
acting on $V$.
These operators induce an action of the Virasoro algebra on $V$ \cite[Proposition~2.7]{FrenkelKac:1980:BasicRepresentationsOfAffineLieAlgebras}.

\section{Virasoro operators and W-algebras}

\section{The Heisenberg double}

\subsection{Positive self-adjoint Hopf algebras}

\begin{Definition}
    A \emph{positive self-adjoint Hopf algebra}, or \emph{PSH}, is a connected graded Hopf algebra over $\ZZ$ with an inner product and a distinguished finite orthogonal $\ZZ$-basis in each grade such that multiplication and comultiplication are adjoint and take elements with positive coefficients in the basis to elements with positive coefficients in the basis.
\end{Definition}

\subsubsection{Symmetric functions}

We let
\[
    \Lambda = \varprojlim_{n \to \infty} k[x_1,\dotsc,x_n]^{\SymGrp n}
\]
be the algebra of symmetric functions\footnote{\url{https://mathoverflow.net/a/85990} contains many references. See also \url{https://www.math.upenn.edu/~peal/polynomials}}.
We have isomorphisms
\[
    \Lambda \cong 
    \bigoplus_{n=0}^\infty Z(\CC[\SymGrp n]) \cong
    \bigoplus_{n=0}^\infty K_0(\catModules{\CC[\SymGrp n]}),
\]
where $Z(\CC[\SymGrp n])$ is the center of the group ring of $\SymGrp n$ and $K_0(\catModules{\CC[\SymGrp n]})$ is the Grothendieck ring of finite dimensional $\SymGrp n$-representations.

The algebra $\Lambda$ has several well-known generating sets:
\begin{itemize}
    \item The \emph{elementary symmetric functions} 
        \[
            e_n = \sum_{i_1 < \dots < i_n} x_{i_1} \dotsm x_{i_n}.
        \]
        Thus $e_0 = 1$, $e_1 = x_1 + x_2 + \dots$, $e_2 = x_1x_2 + x_1x_3 + \dots + x_2x_3  + \dots$.
    \item The \emph{complete symmetric functions} 
        \[
            h_n = \sum_{i_1 \le \dots \le i_n} x_{i_1} \dotsm x_{i_n}.
        \]
        Thus $h_0 = 1$, $h_1 = x_1 + x_2 + \dots$, $h_2 = x_1^2 + x_1x_2 + x_1x_3 + \dots + x_2^2 + x_2x_3  + \dots$.
    \item For any partition $\lambda$ define the \emph{symmetric monomial}
        \[
            m_\lambda = \sum_{\alpha} x^\alpha,
        \]
        where $\alpha$ runs through all multiindices conjugate to $\lambda$ under the $\SymGrp\infty$-action.
        The symmetric monomials generate $\Lambda$ as a free $k$-module.
    \item For any partition $\lambda$ define the \emph{Schur function} (or \emph{S-function})
        \[
            s_\lambda = \sum_T x^{\operatorname{wt} T},
        \]
        where $T$ runs through all semistandard\footnote{A Young tableau is \emph{semistandard} if its entries increase weakly along the rows and strictly along the columns.} Young tableaux $T$ with $\ZZ_{>0}$-entries of shape $\lambda$, and $\operatorname{wt}(T)$ is the weight vector of $T$, i.e., $\operatorname{wt}(T)_i$ is the number of times $i$ appears in $T$.
        The Schur functions generate $\Lambda$ as a free $k$-module.
    \item The \emph{power sum functions}
        \[
            p_n = \sum_i x_i^n.
        \]
        Note that these generate over $k = \QQ$, but not over $k=\ZZ$.
        For a partition $\lambda = (\lambda_1,\dots,\lambda_k)$ set
        \[
            p_\lambda = p_{\lambda_1}\dotsm p_{\lambda_k}.
        \]
\end{itemize}
On defines a comultiplication on $\Lambda$ by
\[
    \Delta(e_n) = \sum_{i=0}^n e_i \otimes e_{n-i}
\]
and a counit by $\epsilon(e_n) = 0$ for $n \ge 0$ (i.e., projection on the constant term).
The power sum functions are primitive:
\[
    \Delta(p_i) = p_{i} \otimes 1 + 1 \otimes p_i.
\]
From this one can deduce the coalgebra structure (over $\QQ$):
\[
    \Delta(p_{i_1}\dotsm p_{i_n}) =
    \sum_{\{j_1,\dots,j_k\} \sqcup \{\ell_1, \dots, \ell_{n-k}\} = \{i_1,\dots,i_n\}} p_{j_1}\dotsm p_{j_k} \otimes p_{\ell_{1}}\dotsm p_{\ell_{n-k}}.
\] 
Further $\Lambda$ obtains the structure of graded (by degree) Hopf algebra with a antipode $S$ satisfying $S(e_n) = (-1)^n h_n$ and $S(p_n) = -p_n$.

The Hopf algebra $\Lambda$ has a standard inner product, called \emph{Hall inner product}, which is characterized by having the Schur functions as an orthonormal basis:
\[
    \langle s_\lambda, s_\mu\rangle = \delta_{\lambda,\mu}.
\]
The power sum functions also provide an orthogonal basis with
\[
    \langle p_\lambda, p_\mu\rangle = z_\lambda\delta_{\lambda,\mu},
\]
where if $m_i$ is the number of parts of size $i$ in $\lambda$ then $z_\lambda = \prod_i m_i!i^{m_i}$.
% Relations to S_n-reps

\subsection{Heisenberg double}

Given a PSH $A$ and an element $x \in A$, define $m_x$ and $\Delta_x$ in $\End(A)$ by
\[
    m_x = m \circ i_x \qquad \Delta_x = j_x \circ \Delta,
\]
where
\[
    i_x(y) = x \otimes y \qquad j_x(y_1 \otimes y_2) = y_x \langle x, y_1 \rangle.
\]
One immediately checks that $m_x$ and $\Delta_x$ are adjoint.
Define a map $\phi\colon A \otimes A \to \End_\ZZ(A)$ by
\[
    x \otimes y \mapsto m_x\Delta_y.
\]
The morphism $\phi$ is injective and its image is a subalgebra of $\End_\ZZ(A)$, called the \emph{Heisenberg double $\Heis(A)$ of $A$} \cite[Proposition~5.4]{GalGal:2017:SymmetricSelfajointHopfCategories}.

\begin{Example}
    Let $A = \Lambda$ be the Hopf algebra of symmetric functions.
    Then $\Lambda$ with the Hall inner product and Schur functions as basis is a PSH [proof?].
    The Heisenberg double $\Heis(\Lambda)$ is the infinite dimensional Heisenberg algebra.
    Indeed setting $a_n = \phi(1 \otimes p_n)$ and $a_{-n} = \phi(p_n \otimes 1)$ for $n \ge 0$ one computes that $[a_i, a_j] = i\delta_{i,-j}$.
\end{Example}

\section{Quantum versions}

\section{Vertex algebras}

\subsection{Formal distributions}

Let $\CC\lParen z,w\rParen$ be the fraction field of the ring of formal power series $\CC\lBrack z,w \rBrack$.
One observes that this is not the same as the ring of formal Laurant series $\sum_{i > N_z}\sum_{j > N_w} a_{i,j} z^iw^i$.
Indeed the element $(z-w)^{-1}$ does not have a unique series representation in $\CC\lBrack z^{\pm1},\, w^{\pm1}\rBrack$: one can expand it as
\[
    i_{z,w}\frac{1}{z-w} \coloneqq \frac1z\sum_{n \ge 0} z^{-n}w^n
\]
or
\[
    i_{w,z}\frac{1}{z-w} \coloneqq -\frac1w\sum_{n \ge 0} z^{n}w^{-n}.
\]
Analytically, the former converges for $|z| > |w|$, while the latter converges for $|w| > |z|$.

\begin{Definition}
    Let $z$ be a (multi-)variable and $R$ a $\CC$-algebra.
    The vector space $R\lBrack z^{\pm1}, w^{\pm1}\rBrack$ is called the \emph{space of formal distributions}.
    The element 
    \[
        \delta(z-w) \coloneqq i_{z,w}\frac{1}{z-w} - i_{w,z}\frac{1}{z-w} = \sum_n z^{-n-1} w^n
    \]
    of 
    $R\lBrack z^{\pm1},\, w^{\pm1}\rBrack$ is called the \emph{formal $\delta$-function.}
\end{Definition}

For $A(z) = \sum_n A_nz^n \in R\lBrack z^{\pm1}, \rBrack$ we set $\Res_z A(z) = A_{-1}$ and $\partial A(z) = \sum_n nA_n z^{n-1}$.

One has an isomorphism $R\lBrack z^{\pm1}\rBrack \to \Hom_\CC(\CC[z^{\pm}], R)$, i.e.~Laurant polynomials can be viewed as test functions on which formal distributions act \cite[Lemma~10.2]{Schottenloher:2008:AMathematicalIntroToCFT}

One notes that $R\lBrack z^{\pm1}\rBrack$ is not a ring, do to potential infinite sums when forming the Cauchy product $A(z)B(z)$ for $A(z), B(z) \in R\lBrack z^{\pm1}\rBrack$.
However, we can form $A(z)B(w) \in R\lBrack z^{\pm1},\, w^{\pm1}\rBrack$.
Further the product of $\delta(z-w)$ with any element of $R\lBrack z^{\pm1}\rBrack$ is always well defined.
Setting $D^j \coloneqq D^j_w \coloneqq \frac{1}{j!} \partial_w^j$ we have
\begin{align*}
    \delta(z-w) & = \delta(w-z) \\
    f(z)\delta(z-q) &= f(w)\delta(z-w) \\
    \Res_z f(z)\delta(z-w) &= f(w) \\
    (z-w)^{n+1} D^n \delta(z-w) & = 0 \quad \text{for all } n \ge 0.
\end{align*}

\begin{Proposition}[{\cite[Proposition~10.6]{Schottenloher:2008:AMathematicalIntroToCFT}}]
    For any $f(z,w) \in R\lBrack z^{\pm1},\, w^{\pm1}\rBrack$ and $N \ge 0$ the following are equivalent:
    \begin{enumerate}
        \item $(z-w)^N f(z,w) = 0$.
        \item $f(z,w) = \sum_{j=0}^{N-1} c^j(w) D_w^j\delta(z-w)$ for some $c^j(w) \in R\lBrack w^{\pm1}\rBrack$.
    \end{enumerate}
    Moreover, in this case $c^j(w)$ is uniquely determined and equal to $\Res_z(z-w)^jf$.
\end{Proposition}

\begin{Definition}
    Two formal distributions $A, B \in R\lBrack z^{\pm1}\rBrack$ are \emph{local} with respect to each other if there exists $N \in \NN$ such that
    \[
        (z-w)^N[A(z), B(w)] = 0
    \]
    in $ R\lBrack z^{\pm1},\, w^{\pm1}\rBrack$. (Note that we do not assume $R$ to be commutative.)
\end{Definition}

In the context of formal distributions for $A(z) = \sum_n A_n z^n$ one sets $A_{(n)} = A_{-n-1} = \Res_z(A(z)z^{n})$ and
\[
    A(z)_- = \sum_{n \ge 0} A_{(n)}z^{-n-1}, \qquad A(z)_+ = \sum_{n < 0} A_{(n)} z^{-n-1},
\]
so that $A(z) = A(z)_{-} + A(z)_+$.

\begin{Definition}
    The \emph{normal ordered product} of two formal distributions $A, B \in R\lBrack z^{\pm1}\rBrack$ is
    \[
        \normord{A(z)B(w)} \coloneqq A(z)_+B(w) + B(w)A(z)_- \in  R\lBrack z^{\pm1},\, w^{\pm1}\rBrack.
    \]
\end{Definition}

Note that in the normal order product all coefficients of $z^n$ in $A$ and $B$ with $n$ positive are multiplied from the left with coefficients for $n < 0$.


Let
\[
    i_{z,w}\frac{1}{(z-w)^{j+1}} \coloneqq \sum_{n\ge 0} \binom{n}{j} z^{-n-1}w^{n-j}
\]
and
\[
    i_{w,z}\frac{1}{(z-w)^{j+1}} \coloneqq -\sum_{n\ge 1} \binom{-n}{j} z^{n-1}w^{-n-j}
\]
be the expansions for $|z| > |w|$ and $|w| > |z|$ respectively.


\begin{Proposition}[{\cite[Theorem~10.10]{Schottenloher:2008:AMathematicalIntroToCFT}}]
    For $A, B \in R\lBrack z^{\pm1}\rBrack$ and $N \ge 0$ the following are equivalent:
    \begin{enumerate}
        \item $A$ and $B$ are local with respect to each other with $(z-w)^N f(z,w) = 0$.
        \item $[A(z), B(w)] = \sum_{j=0}^{N-1} C^j(w) D^j\delta(z-w)$ for some $C^j(w) \in R\lBrack  w^{\pm1}\rBrack$.
        \item 
            $A(z)B(w) = \sum_{j=1}^{N-1} i_{z,w}\frac{1}{(z-w)^{j+1}} C^j(w) + \normord{A(z)B(w)}$ and
            $B(w)A(z) = \sum_{j=1}^{N-1} i_{w,z}\frac{1}{(z-w)^{j+1}} C^j(w) + \normord{A(z)B(w)}$
            for some $C^j(w) \in R\lBrack  w^{\pm1}\rBrack$.
    \end{enumerate}
\end{Proposition}

In the above case one calls $\normord{A(z)B(w)}$ the regular part of $A(z)B(w)$ and $\sum_{j=1}^{N-1} i_{z,w}\frac{1}{(z-w)^{j+1}} C^{j(w)}$ the singular part and writes
\[
    A(z)B(w) \sim \sum_{j=1}^{N-1} i_{z,w}\frac{1}{(z-w)^{j+1}} C^j(w)
\]
Sometimes one leaves out the $i_{z,w}$, silently assuming $|z| > |w|$ and we may do so below.

\begin{Example}
    Let $R$ be the infinite dimensional Heisenberg algebra with generators $a_n$, $n \in \ZZ$ and $c$, and relations $[a_m, a_n] = \delta_{m,-n}c$ and $[a_n, c] = 0$.
    Set $A(z) = \sum_{n \in \ZZ} z^{-n-1}$.
    One computes
    \[
        [A(z), A(w)] = \sum_{m, n} [a_m, a_n] z^{-m-1}w^{-n-1} = \sum_m m z^{-m-1}w^{m-1}x = \partial_w\delta(z-w)C.
    \]
    Thus $A$ is local with respect to itself:
    \[
        A(z)A(w) \sim \frac{c}{(z-w)^2}.
        \qedhere
    \]
\end{Example}

\begin{Example}
    Let $R$ be the Virasoro algebra with generators $L_n$, $n \in \ZZ$ and $c$, and relations $[L_m,L_n] = (m-n)L_{m+n} \frac{m}{12}(m^2-1)\delta_{m,-n}c$ and $[L_n, C] = 0$.
    Set $L(z) = \sum_{n \in \ZZ} z^{-n-2}$.
    Then $L$ is local with respect to itself:
    \[
        [L(z), L(w)] =
        \frac{c}{12} \partial_w^3\delta(z-w) + 2L(w)\partial_w\delta(z-w) + (\partial_wL(w))\delta(z-w),
    \]
    and
    \[
        L(z)L(w)] \sim
        \frac{c}{2} \frac{1}{(z-w)^4} + \frac{2L(w)}{(z-w)^2} + \frac{\partial_wL(w)}{z-w}.
        \qedhere
    \]
\end{Example}

\subsection{Fields}

\begin{Definition}
    Let $V$ be a complex vector space.
    An element
    \[
        a = \sum_n a_{(n)} z^{-n-1} \in \End_\CC(V)\lBrack z^{\pm1} \rBrack
    \]
    is a \emph{field} if for all $v \in V$ there exists $n_0 \in \NN$ such that $a_{(n)} \cdot v = 0$ for all $n \ge n_0$.
    We write $\mathcal F(V)$ for the vector space of all fields on $V$.
\end{Definition}

\begin{Example}\label{ex:heisenberg_field}
    Let $V = k[x_1,x_2,\dots]$ be the polynomial ring in infinitely many variables and $\rho\colon H \to \End(V)$ the Fock space representation:
    for $n > 0$ we have
    \begin{align*}
        \rho(a_n)(P) &= \frac{\partial}{\partial x_n}P, \\
        \rho(a_0)(P) & = 0 \\
        \rho(a_{-n)}(P) & = nx_nP \\
        \rho(c)(P) & = P
    \end{align*}
    One immediately checks that
    \[
        \Phi(z) = \sum_n \rho(a_n) z^{-n-1} \in \End(V)\lBrack z^{\pm1} \rBrack
    \]
    is a field on $V$.
\end{Example}

[Add \cite[Proposition 10.17]{Schottenloher:2008:AMathematicalIntroToCFT}?]

\begin{Definition}
    A \emph{vertex algebra} is a vector space $V$ with a distinguished vector $\Omega$ (or $|0\rangle$), called the \emph{vacuum vector}, an endomorphism $T \in \End(V)$, called \emph{infinitesimal translation operator}, and a linear map $Y \colon V \to \mathcal{F}(V)$,
    \[
        a \mapsto Y(a, z) = \sum_{n \in \ZZ} a_{(n)} z^{-n-1}, \quad a_{(n)} \in \End(V),
    \]
    called the \emph{vertex operator} and providing the \emph{state field correspondence}, such that
    \begin{description}
        \item[V1 (translation invariance)] $[T,\, Y(a,z)] = \partial Y(a,z)$,
        \item[V2 (locality)] For any $a, b \in V$, $Y(a,z)$ and $Y(b,z)$ are local,
        \item[V3 (vacuum)] $T\Omega = 0$, $Y(\Omega, z) = \id_V$ and $\res{Y(a,z)\Omega}{z=0} = a$.
    \end{description}
\end{Definition}

\begin{Theorem}[{\cite[Theorem~10.24]{Schottenloher:2008:AMathematicalIntroToCFT}}]
    Let $V$ be a vector space with an endomorphism $T$ and a distinguished vector $Ω∈V$.
    Let $(Φ_a)_{a∈I}$ be a collection of fields $Φ_a(z) = ∑a_{(k)}z^{−k−1} = a(z) ∈ \End(V)\lBrack z^{±1}\rBrack$, indexed by a linear independent subset $I⊂V$ such that the following conditions are satisfied for all $a,b∈I$:
    \begin{enumerate}
        \item $[T,Φ_a(z)] = ∂Φ_a(z)$
        \item $TΩ = 0$ and $\res{Φ_a(z)Ω}{z=0} =a$.
        \item $Φ_a$ and $Φ_b$ are local with respect to each other for all $a, b \in I$
        \item The set $\Bigl\{a^1_{(−k_1)} a^2_{(−k_2)} \dotsm a^n_{(−k_n)} \cdot Ω : a^j ∈ I,\, k_j>0 \Bigr\}$ of vectors along with $Ω$ forms a basis of $V$.
    \end{enumerate}
    Then there exists a unique vertex algebra with translation operator $T$, vacuum vector $Ω$, and $Y(a,z)=Φ_a(z)$ for all $a∈I$, given by
    \[
        Y\bigl(a^1_{(−k_1)} \dotsm a^n_{(−k_n)} \cdot Ω,\, z\bigr) \coloneqq
        \normord{D^{k_1−1}Φ_{a^1}(z) \dotsm D^{k_n−1}Φ_{a_n}(z)}
    \]
    together with $Y(Ω,z) = \id_V$.
\end{Theorem}

[I am confused here: how to make sense of expressions like $\normord{\Phi_a(z)\Phi_b(z)}$ without infinite sums?]

\begin{Example}
    Continuing Example~\ref{ex:heisenberg_field}, the element $1$ generates $V = \CC[x_1,\dots]$ under the Heisenberg action.
    Thus setting 
    \begin{align*}
        \Phi_1(z) & = \Phi(z) = \sum_n \rho(a_n) z^{-n-1}, \\
        \Omega &= 1, \\
        T &= \sum_{m > 0} \rho(a_{-m-1})\rho(a_m)
    \end{align*}
    gives the \emph{Heisenberg vertex algebra.}
    We note that $T$ is uniquely determined by the conditions $T\Omega = 0$ and $[T, a_n] = -na_{n-1}$.
\end{Example}

\printbibliography

\end{document}
