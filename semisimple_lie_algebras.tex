%type: notes
%title: Semisimple Lie algebras
%tags: Lie theory, Lie algebra, semisimple Lie algebra, sl(2)
\documentclass[english]{short-notes}

\usepackage{math-alg,math-ag}

\addbibresource{global.bib}
%\bibliography{global.bib}


\title{Some notes on semisimple Lie algebras}
\author{Clemens Koppensteiner}

\renewcommand\dual{*}
\newcommand\dualroot[1]{#1^{∨}}

\begin{document}

\maketitle

\bigskip

We work over the base field $ℂ$.
The notation is based on \cite[Chapter~9]{HottaTakeuchiTanisaki:2008:DModulesPerverseSheavesRepresentationTheory}.
The standard reference for Lie algebras is \cite{Humphreys:1978:IntroLieAlgebrasAndRepresentationTheory}.

Let $\liealg g$ be a semisimple Lie algebra over $ℂ$.

\begin{Def}
The \emph{Killing form} of $\liealg g$ is
\[
κ(x,y)\colon \liealg g × \liealg g → ℂ,\ κ(x,y) = \trace(\ad(x)\ad(y)).
\]
It is a non-degenerate symmetric bilinear form.

A \emph{Cartan subalgebra} $\liealg h$ of $\liealg g$ is a subalgebra that is maximal among all subalgebras satisfying the following conditions
\begin{itemize}
    \item $\liealg h$ is commutative;
    \item for every element $h ∈ \liealg h$, the linear transformation $\ad(h)$ is semisimple (i.e.\ every invariant subspace has a complementary invariant subspace; equivalently: $\ad(h)$ is diagonalizable.
\end{itemize}

If $\liealg h₁$ and $\liealg h₂$ are Cartan subalgebras, then there exists some $g ∈ \Aut(\liealg g)$ such that $g(\liealg h₁) = \liealg h₂$.
From now on we fix a Cartan subalgebra $\liealg h$ of $\liealg g$.

For each $α ∈ \liealg h^\dual$ we set
\[
\liealg g_α = \left\{ x ∈ \liealg g : \ad(h)x = α(h)x \text{ for all } h ∈ \liealg h\right\}.
\]
Then $\liealg g_0 = \liealg h$ and each other $\liealg g_α$ is either $0$ or $1$-dimensional and $\liealg g = \bigoplus_{α ∈ \liealg h^\dual} \liealg g_α$.
The \emph{root system} of $\liealg g$ (with respect to $\liealg h$) is
\[
\Delta = \left\{ α ∈ \liealg h^\dual \setminus \{0\}  : \liealg g_α \ne 0 \right\}.
\]
Then $Δ$ is a root system in $\liealg h^\dual$.
In particular, this means that for each root $α ∈ Δ ⊆ \liealg h$ there exists a unique dual root $\dualroot α ∈ \liealg h$ such that $α(\dualroot α) = 2$ and the reflection
\[ s_α\colon β \mapsto β - β(\dualroot α) β \]
maps $Δ$ to itself.

We fix a subset $Δ^+$, called \emph{positive roots} of $Δ$ such that $Δ^+ ∪ (-Δ^+) = Δ$, $Δ^+ ∩ (-Δ^+) = \emptyset$ and if $α,β ∈ Δ^+$ with $α + β ∈ Δ$, then $α + β ∈ Δ^+$.
Any positive root that cannot be written as the sum of two positive root is called a \emph{simple root}.
Let $Π$ denote the set of simple roots. 
It is a basis for $\liealg h^\dual$.

The group
\[
W = \gen{ s_α : α ∈ Δ } = \gen{ s_π : π ∈ Π}
\]
is called the \emph{Weyl group}.
We have $W(Π) = Δ$.

The \emph{root lattice} is the lattice
\[
Λ_r = \sum_{α ∈ Δ} ℤα = \bigoplus_{π ∈ Π} ℤα_i.
\]
Inside this, we have the \emph{dominant roots}
\[
Λ_r^+ = \sum_{α ∈ Δ^+} \mathbb N α = \bigoplus_{π ∈ Π} \mathbb N α_i,
\]
where $\mathbb N = \{0,1,2,\dotsc\}$.
Let $Π = \{ α₁,\dotsc, α_n\}$.
Define the \emph{fundamental weights} $λ_i$ by the equation $λ_i(\dualroot{α_j}) = δ_{ij}$.
We obtain the \emph{weight lattice}
\[
Λ = \{ λ ∈ \liealg h^\dual : λ(\dualroot α) ∈ ℤ \text{ for all } α ∈ Δ\} = \bigoplus_{i=1}^n ℤλ_i
\]
and the dominant weights
\[
Λ^+ = \{ λ ∈ \liealg h^\dual : λ(\dualroot α) ∈ \mathbb N \text{ for all } α ∈ Δ^+\} = \bigoplus_{i=1}^n \mathbb N λ_i.
\]
Define a partial ordering $≤$ on $\liealg h^\dual$ by $α≤β$ if $β-α ∈ Λ_r^+$.
\end{Def}

\begin{Thm}
    Any finite-dimensional representation of a semisimple Lie algebra $\liealg g$ is completely reducible.
\end{Thm}

For a finite dimensional representation $σ\colon \liealg g → \lieglof V$ and $α ∈ \liealg h^\dual$
\[
V_α = \left\{ v ∈ V : σ(h)(v) = α(h)v \text{ for all } h ∈ \liealg h \right\}.
\]
Then $V = \bigoplus_{α ∈ \liealg h^\dual} V_α$.
If $V_α \ne 0$, then $α$ is called a \emph{weight} of $V$ and $V_α$ is the \emph{weight space} of $V$ with weight $λ$.
Every weight is an element of the weight lattice $Λ$.

\begin{Thm}
    Let $V$ be a finite-dimensional irreducible $\liealg g$-module.
    Then $V$ has a unique maximal weight (with respect to the partial order on $\liealg h^\dual$); it is called the \emph{highest weight} of $V$.

    Conversely, if $λ ∈ Λ^+$ is a positive weight, then there exists a unique (up to isomorphism) irreducible $\liealg g$-module with highest weight $λ$.
\end{Thm}

The set $\dualroot Δ = \{ \dualroot α : α ∈ Δ\}$ is a root system in $\liealg h$.
The Weyl groups of $Δ$ and $\dualroot Δ$ are naturally identified via $s_α \longleftrightarrow s_{\dualroot α}^{-1}$.
Hence $W$ acts on $\liealg h$.

\begin{Def}
    The algebra $\liealg z = (\SymAlg \liealg h)^W$ is called the \emph{Harish-Chandra center}.
    It is isomorphic (as algebras) to the center of the universal enveloping algebra of $\liealg g$.

    Any algebra homomorphism $\liealg z → ℂ$ is called a \emph{central character}.
    We can identify $\SymAlg \liealg h$ with the algebra of polynomial functions on $\liealg h^\dual$.
    Using this we define, for each $α ∈ \liealg h^\dual$, a central character $χ_α\colon \liealg z → ℂ$ by $χ_α(z) = z(α)$.
\end{Def}

\begin{Thm}
    Every central character is of the form $χ_α$ for some $α ∈ \liealg h^\dual$.
    For $α,β ∈ \liealg h^\dual$ the central characters $χ_α$ and $χ_β$ are equal if and only if $α$ and $β$ are in the same $W$-orbit.
\end{Thm}

In this way we get an action of $\liealg z$ on any representation of $\liealg g$: on an irreducible representation with highest weight $λ$, we let $z ∈ \liealg z$ act by multiplication with $χ_λ(z)$.

\subsection*{\texorpdfstring{$\liesl2ℂ$}{sl2C}}

The Lie algebra $\liesl2ℂ$ consists of all traceless $2×2$-matrices.
The standard basis is
\[
E = \begin{pmatrix} 0 & 1 \\ 0 & 0 \end{pmatrix}, \quad
F = \begin{pmatrix} 0 & 0 \\ 1 & 0 \end{pmatrix}, \quad
H = \begin{pmatrix} 1 & 0 \\ 0 & -1 \end{pmatrix},
\]
so that
\[
[E,F] = H, \quad
[H,E] = 2E, \quad
[H,F] = -2F.
\]
We choose a Cartan subalgebra
\[
\liealg h = Hℂ = \left\{h_a = \begin{psmallmatrix} a & 0 \\ 0 & -a\end{psmallmatrix} : a ∈ ℂ \right\}.
\]
For $b ∈ ℂ$, we denote the element $h_a ↦ ab$ of $\liealg h^\dual$ by $α_b$.
From the commutator relations above, we see that $Eℂ = \liealg g_2$ is an eigenspace of $\ad(H)$ for the eigenvalue $2$ and $Fℂ$ is one for $-2$.
Hence,
\[
Δ = \{ α_{-2}, α_2\} \text{ with }
\dualroot{α_{-2}} = \begin{psmallmatrix} -1 & 0 \\ 0 & -1\end{psmallmatrix} \text{ and }
\dualroot{α_{2}} = \begin{psmallmatrix} 1 & 0 \\ 0 & 1\end{psmallmatrix}.
\]
We compute that
\[
s_{α_2} = s_{α_{-2}}\colon α_b \mapsto -α_b = α_{-b}.
\]
Hence the Weyl group $W$ is isomorphic to $\rquot ℤ2$ acting by multiplication with $-1$.
We set $Δ^+ = Π = \{ α_2\}$. 
The fundamental weight is $α₁$.
In the following picture the weight lattice $Λ$ is marked with dots and the root lattice $Λ_r$ is marked with circles.
\[
\begin{tikzpicture}
    \draw[gray] (-4.5,0) -- (4.5,0) node[right] {$\liealg h^\dual$};
    \foreach \i in {-4,-3,...,4} {
        \fill (\i,0) circle [radius=0.5ex] node[above=1ex] {$\i$};
    }
    \foreach \i in {-4,-2,...,4} {
        \draw (\i,0) circle [radius=1.0ex];
    }
    \draw[dashed] (0,-1) -- (0,1.5) node[above=-1.5ex,font=\small] {$s_{α₂} = s_{α_{-2}}$};
    \draw[<->] (-0.5,1) -- (0.5,1);
\end{tikzpicture}
\]
One easily checks that $W \cong \rquot ℤ2$ acts on $\liealg h$ also via multiplication by $-1$.
The isomorphism $\liealg h \cong ℂ$ yields an isomorphism $\SymAlg \liealg h \cong \SymAlg ℂ \cong ℂ[t]$. 
On $ℂ[t]$, the Weyl group acts by $t \mapsto -t$.
Hence we have
\[
\liealg z = (\SymAlg \liealg h)^W \cong ℂ[t²].
\]
Let $V$ be an irreducible representation with highest weight $λ = α_b$. 
Tracing through the identifications, we see that $f(t) ℂ[t²] \cong \liealg z$ acts on $V$ by multiplying with $f(b)$.
If we further identify $ℂ[t²]$ with $ℂ[t]$, this becomes multiplying with $f(b²)$.
We can now see $V$ as coherent sheaf over $\as 1$: it is the skyscraper sheaf with stalk $V$ supported at $λ²$.

Alternatively, $\liealg z$ acts on $\End(V) \cong ℂ$ in the same way and we can see $\End(V)$ as the $1$-dimensional skyscraper supported at $λ²$.
With this interpretation, for an arbitrary representation $V$, we can read of the irreducible components and their multiplicity by where the sheaf is supported and what dimensions the stalks have.


\printbibliography
\end{document}
