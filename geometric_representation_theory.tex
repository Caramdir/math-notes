%type: notes
%title: geometric representation theory
%tags: representation theory, algebraic groups, geometric representation theory
\documentclass[english, no-theorem-numbers]{short-notes}

\usepackage{math-alg,math-ag}

\addbibresource{global.bib}
%\bibliography{global.bib}

\title{basic theorems of geometric representation theory}
\author{Clemens Koppensteiner}

\begin{document}
\maketitle

\section{notation}

We fix a maximal torus $H$ and a Borel $B$ containing $H$.

We write $X^*(G) = X^*(H) = \Hom(H,k^\units)$.

We set $X = \mathcal{Fl} = \rquot GB$.

\section{borel--weil--bott}

\paragraph{References}
    \cite[Section~2.5]{Schmid:2005:GeometricMethodsInRepresentationTheory},
    \cite[Section~9.11]{HottaTakeuchiTanisaki:2008:DModulesPerverseSheavesRepresentationTheory},
    \href{http://en.wikipedia.org/wiki/Borel–Weil–Bott_theorem}{Wikipedia}

\begin{Lem}
    $G$-equivariant vector bundles (i.e.\ $G$-equivariant coherent sheaves) on $X$ are the same as $B$-modules.
\end{Lem}

The correspondence sends an equivariant vector bundle to its fiber over $B ∈ X$.
Conversely, a $B$-modules $V$ is sent to $\lquot{B}{(G×V)}$.

Consider the Cartesian square
\[
    \begin{tikzcd}
        \rquot GB \arrow{d}{p} \arrow{r}{f} \arrow{d} & BB \arrow{d}{a}\\
        \pt \arrow{r} & BG
    \end{tikzcd}
\]
\begin{Claim}
    The equivalence is given by the functor $f^!\colon \catCoh{BB} → \catCoh{X}^G$.
\end{Claim}

If we restrict to line bundles, than we see that a one-dimensional representation of $B$ is the same a one-dimensional representation of $H$ (since $R(B)$ has to act trivially), i.e.~a character of $H$.
We write $\mathcal L_λ$ for the line bundle on $X$ corresponding to $λ ∈ X^*(H)$.
In other words, we have $\mathcal L_λ = f^!b^!ℂ_λ$, where $b\colon BB→BH$ is induced by the quotient map $B→H$ and $ℂ_λ ∈ \catRep H$ is the representation corresponding to $λ ∈ X^*(H)$.

The (derived) global sections of a vector bundle on $X$ (or any projective variety with a $G$-action) give a $G$-representation.
In other words we have the functor $p_*\colon \catCoh{G/B}^G → \catCoh{\pt}^G \cong \catCoh{BG}$.
This gives a way to seeing $G$-representations in a geometric context.

We are interested in describing the functor $\operatorname{ind} = a_*b^! \cong p_*f^!b^!\colon \catCoh{BH} → \catCoh{BG}$.

\begin{Thm}[Borel--Weil--Bott]
    If $λ$ is singular (i.e.~there exists no $w ∈ W$ such that $w(λ+ρ)-ρ$ is dominant), then $\operatorname{ind}(ℂ_λ) = 0$.
    Otherwise, let $w$ be the unique element of the Weyl group such that $w(λ+ρ)-ρ$ is dominant.
    Then $\operatorname{ind}(ℂ_λ) = V_{w(λ+ρ)-ρ}[-\ell(w)]$, where $V_χ$ is the irreducible representation of highest weight $χ$.
\end{Thm}

Note that $\mathcal L_{-2ρ}$ is the canonical bundle on $X$ so that this description is compatible with Serre duality.

\section{bb-localization}

\section{bgg-resolution}

\printbibliography
\end{document}
