%type: notes
%title: The nilpotent cone of sl(2)
%tags: sl(2), symplectic variety, cone, nilpotent cone, Poisson algebra
\documentclass[english]{short-notes}

\usepackage{math-alg,math-ag}

\addbibresource{global.bib}
%\bibliography{global.bib}


\title{the nilpotent cone of \texorpdfstring{$\liesl2ℂ$}{sl2C}}
\author{Clemens Koppensteiner}

\begin{document}

\newcommand\D{\mathrm{d}}

\maketitle

\section{the poisson structure}
The nilpotent cone of $\liesl2ℂ$ is 
\[
    N = \Spec \rquot{ℂ[x,y,z]}{(x²+4yz)}.
\]
The Poisson structure is given by
\[
    \{x,y\} = 2y, \quad \{x,z\} = -2z, \quad \{y,z\} = x.
\]
To check that this is well-defined, one needs to compute that $\{x²+4yz,\, g(x,y,z)\} = 0$ for all $g$.

Alternatively, if $N$ is given by the equation $x²+yz$, then one needs to set $\{y,z\} = \frac14 x$.
If $N$ is given by $x²-yz$, then $\{y,z\} = -\frac14x$.

Another way to give $N$ is 
\[
    N = \Spec ℂ[\frac12x²,\, -xy, -\frac12y²].
\]
Then the Poisson bracket is defined by
\[
    \{-xy, \frac12x²\} = x², \quad
    \{-xy, -\frac12y²\} = y², \quad
    \{\frac12x², -\frac12y²\} = -xy.
\]

Note that in particular $HP_0(X) = \rquot{\O_N}{\{\O_N,\O_N\}} = ℂ$.

\section{(co)tangent space and anchor map}

We use the first equation above, i.e.~set $N = \Spec \rquot{ℂ[x,y,z]}{(f)}$ with $f=x²+4yz$.
The cotangent complex of $N$ is given by
\[
    \mathbb L_N = 
    \bigl( \O_N \xrightarrow{\begin{psmallmatrix}∂_xf\\∂_yf\\∂_zf\end{psmallmatrix}} \O_N³\bigr) = 
    \bigl( \O_N \xrightarrow{\begin{psmallmatrix}2x\\4z\\4y\end{psmallmatrix}} \O_N³\bigr).
\]
Hence the cotangent sheaf is 
\[
    Ω¹_N = H^0(\mathbb L_N) = \rquot{\left( \O_N \D x + \O_N\D y + \O_N \D z \right)}{(2x \D x,\, 4z\D y,\, 4y \D z)}.
\]
The Poisson tensor $Ω ∈ \sheaf T_N ∧ \sheaf T_N$ induces an $\O_N$-module map
\[
    ω\colon Ω¹_N → \sheaf T_N\colon g\D f \mapsto g\{f,\,{-}\}.
\]
Explicitly, the computations above show that
\begin{align*}
    \D x & \mapsto  2y∂_y - 2z∂_z, \\
    \D y & \mapsto -2y∂_x + x∂_z, \\
    \D z & \mapsto  2z∂_x - x∂_y.
\end{align*}
Note that all of these vanish at the origin. 
Further, one computes that 
\[
    ω(\D f) = ω(2x\D x + 4z\D y + 4y \D z) = 0,
\]
so that $ω$ (trivially) extends to a map of complexes $ω\colon \mathbb L_N → \mathbb T_N$ (with $0$ maps in non-zero degrees).

%\printbibliography
\end{document}
