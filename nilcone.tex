%type: notes
%title: The nilpotent cone of sl(2)
%tags: sl(2), symplectic variety, cone, nilpotent cone, Poisson algebra
\documentclass[english]{short-notes}

\usepackage{math-alg,math-ag}

\addbibresource{global.bib}
%\bibliography{global.bib}


\title{The nilpotent cone of \texorpdfstring{$\liesl2ℂ$}{sl2C}}
\author{Clemens Koppensteiner}

\begin{document}

\maketitle

\noindent The nilpotent cone of $\liesl2ℂ$ is 
\[
    N = \Spec \rquot{ℂ[x,y,z]}{(x²+4yz)}.
\]
The Poisson structure is given by
\[
    \{x,y\} = 2y, \quad \{x,z\} = -2z, \quad \{y,z\} = x.
\]
To check that this is well-defined, one needs to compute that $\{x²+4yz,\, g(x,y,z)\} = 0$ for all $g$.

Alternatively, if $N$ is given by the equation $x²+yz$, then one needs to set $\{y,z\} = \frac14 x$.
If $N$ is given by $x²-yz$, then $\{y,z\} = -\frac14x$.

Another way to give $N$ is 
\[
    N = \Spec ℂ[\frac12x²,\, -xy, -\frac12y²].
\]
Then the Poisson bracket is defined by
\[
    \{-xy, \frac12x²\} = x², \quad
    \{-xy, -\frac12y²\} = y², \quad
    \{\frac12x², -\frac12y²\} = -xy.
\]

Note that in particular $HP^0(X) = \rquot{\O_N}{\{\O_N,\O_N\}} = ℂ$.

%\printbibliography
\end{document}
