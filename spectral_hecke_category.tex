%type: notes
%title: Spectral Hecke category
%tags: representation theory, geometric Langlands
\documentclass[english]{short-notes}

\usepackage{math-alg,math-ag, math-gl}
\usepackage{xparse}

\addbibresource{global.bib}
%\bibliography{global.bib}

\newcommand\SingSupp{\operatorname{SingSupp}}
\newcommand\KD{\operatorname{KD}}

\title{the spectral hecke category}
\author{Clemens Koppensteiner}


\begin{document}

\maketitle

\begin{Def}
    The \emph{spectral Hecke stack} is the dg stack
    \[
        \sHecke G = \sHeckeStack G.
    \]
    The \emph{spectral Hecke category} is $\sHeckeCat G$.
\end{Def}

Note that the Koszul dual description of the singular support of applies to $\sHecke G$.
Hence its singular support stack is $\Sing(\sHecke G) = \rquot{\liealg g^*}G$.

\begin{Q}
    How does $\sHecke G$ compare to $\Loc_G(\ps 1)$?
\end{Q}

\begin{Q}
    Can we describe $\sHeckeCat G$ in terms of parabolic induction?
\end{Q}

\begin{Q}
    Can we describe the filtration of $\sHeckeCat G$ by closures of nilpotent orbits?
    What are the corresponding subcategories of $\operatorname{Sph}(G,x)$?
\end{Q}

\section*{koszul duality}

We have a Koszul duality isomorphism
\[
    \KD \colon \catIndCoh{\sHecke G} \isoto \left( \catModules{\SymAlg(\liealg g^*[-2])} \right)^G.
\]

\begin{Q}
    Is $\left( \catModules{\SymAlg(\liealg g^*[-2])} \right)^G \cong \catQCoh{g^*}^{\Gm × G}$?
\end{Q}

\section*{\enquote{naive} geometric satake}

As a \enquote{warm-up} can one answer these questions for the categories in the \enquote{naive} geometric Satake
\[
    \operatorname{Rep}(\ld G) \cong \cat{Perv}(\mathrm{Gr}_G)^{G(\O)}?
\]

\section*{relation between \texorpdfstring{$BG$ and $\sHecke G$}{BG and the spectral Hecke stack}}

Note that we have a diagonal map $Δ\colon BG → \sHeckeStack G$ that induces a (monoidal?) map $Δ_*\colon \catRep G → \catIndCoh{\sHecke G}$.
Since the map $BG → \rquot{\liealg g}{G}$ is separated (by same reference in the stacks project), $Δ$ is a closed immersion (and in particular proper).

\subsection*{monadic description}

The stack $BG$ is smooth and hence $\Sing{BG} = BG$ and $\catIndCoh{BG} = \catQCoh{BG}$.
By \cite[Theorem~6.3.3]{ArinkinGaitsgory:arXiv:v2:SingularSupport}, we have
\[
    \SingSupp(Δ_*\sheaf F) = \Sing(\sHecke G) = \rquot{\liealg g^*}{G}.
\]
By \cite[Corollary~7.4.12]{ArinkinGaitsgory:arXiv:v2:SingularSupport}, the functor $Δ_*$ is conservative (hence the essential image of $Δ^*$ generates $\catQCoh{BG}$).
By \cite[Proposition~7.4.19]{ArinkinGaitsgory:arXiv:v2:SingularSupport}, the essential imago of $\catQCoh{BG}$ under $Δ_*$ generates $\catIndCoh{\sHecke G}$.
In particular, $Δ^!$ is conservative.

\begin{Cor}
    We have a monad $Δ^!Δ_*$ on $\catQCoh{BG}$ and a comonad $Δ_*Δ^!$ on $\catIndCoh{\sHecke G}$.
    The (co)modules over these (co)monads describe the respective other category.
\end{Cor}

\begin{Q}
    Can we describe $Δ^!Δ_*$ and $Δ_*Δ^!$?
\end{Q}

\begin{Q}
    How do these monads interact with restriction of singular support to $\rquot{\mathrm{Nilp}(\liealg g^*)}G$?
\end{Q}

\subsection*{eisenstein series}

\begin{Claim}
    There is a commutative diagram
    \[
        \begin{tikzcd}
            BG \arrow{d}{Δ} & BB \arrow{l} \arrow{r} \arrow{d}{Δ} & BH \arrow{d}{Δ} \\
            \sHeckeStack G & \sHeckeStack B \arrow{l}\arrow{r} & \sHeckeStack H
        \end{tikzcd}
    \]
\end{Claim}

\begin{Q}
    Are the squares in this diagram Cartesian?
    If so, can we relate the induction/restriction functors for $BG$ with the Eisenstein series/constant term functors for $\sHecke G$?
\end{Q}

\subsection*{koszul duality}

We have a Kozsul duality isomorphism
\[
    \KD \colon \catIndCoh{\sHecke G} \isoto \left( \catModules{\SymAlg(\liealg g^*[-2])} \right)^G.
\]

\begin{Q}
    Is there a nice description of the functors $Δ_*$, $Δ^!$ combined with $\KD$, i.e.~as functors $\catQCoh{BG}^G \leftrightarrows \catQCoh{\liealg g^*}^{\Gm × G}$?
\end{Q}

\section*{filtration by orbits}

By Koszul duality, we have
\[
    \catIndCoh{\pt ×_{\liealg g} \pt} \cong \catModules{\SymAlg(\liealg g[-2])}.
\]
(In David's notation the second category is $\catIndCoh{(\pt ×_{\liealg g} \pt)^*[1]}$.)
Thus we can localize the category at closed conical subvarieties of $\liealg g^*$; in particular at closures of nilpotent orbits.
The categories associated to nilpotent orbit closures are then generated by the corresponding structure sheaves.

Alternatively, we can use coherent IC sheaves \cite{ArinkinBezrukavnikov:arXiv:PerverseCoherentSheaves}.
What is the relationship between the perverse t-structure on $N$ and the filtration of $\sHeckeCat G$ by nilpotent orbits?

\section*{the categories in the satake equivalence}

\newcommand\Sph{\operatorname{Sph}}
\[
    \begin{tikzcd}
        \sHeckeCat{\ld G} \arrow[yshift=0.4ex]{r} \arrow{d}[description]{\cong} &
        \catIndCoh{\sHecke{\ld G}} \arrow[yshift=-0.4ex]{l} \arrow{d}[description]{\cong}[above=0.5ex, sloped]{\mathrm{Sat}^{\mathrm{ren}}} &
        %
        & \catRep{\ld G} \arrow{ll}{Δ_*} \arrow{d}[description]{\cong}[sloped, above=0.5ex]{\mathrm{Sat}^{\mathrm{naive}}} \arrow{dl}[above,sloped]{\mathrm{Sat}^{\mathrm{naive,ren}}}\\
        %
        %
        \Sph(G,x) \arrow[yshift=0.4ex]{r}{Ξ} &
        \Sph(G,x)^{\mathrm{ren}} \arrow[yshift=-0.4ex]{l}{Ψ} &
        \Sph(G,x)^{\mathrm{naive,ren}} \arrow{l} &
        \Sph(G,x)^{\mathrm{naive}} \arrow{l}
    \end{tikzcd}
\]

\printbibliography
\end{document}
