%type: notes
%title: Second Adjointness
%tags: representation theory
\documentclass[english]{short-notes}

\usepackage{math-alg,math-ag, math-gl}
\usepackage{xparse}

\addbibresource{global.bib}
%\bibliography{global.bib}

\title{second adjointness}
\author{Clemens Koppensteiner}

\renewcommand\B{\mathrm{B}}
\newcommand\Ind{\operatorname{Ind}}
\newcommand\Res{\operatorname{Res}}

\begin{document}

\maketitle

This is an attempt to formulate the \enquote{Second Adjointness Theorem} for reductive groups in geometric terms.

For an algebraic group $G$ let $\catRep G = \catQCoh{\B G} = \catIndCoh{\B G}$ be its representation category.
We assume from now on that $G$ is reductive and consider a parabolic $P$ of $G$ with corresponding Levi $L$.
Then we have maps
\[
    \B G \xleftarrow{p} \B P \xrightarrow{q} \B L
\]
with $p$ schematic, proper and smooth and $q$ smooth (but not schematic).
Hence one can define an induction functor
\[
    \Ind\colon \catRep L → \catRep G
\]
by $\Ind = p_*q^!$.

The functor $q^!$ sends coherent sheaves to coherent sheaves \cite[Lemma~7.1.2]{Gaitsgory:preprint:IndcoherentSheaves} and hence has a right adjoint $q_{\bullet}$.
Specifically, since $q$ is smooth we have an isomorphism of functors $q^! \cong q^!(\O_{\B L}) \otimes q^*$.
\begin{Claim}
    We have $q^!(\O_{\B L}) = \O_{\B P}$ and hence $q_\bullet = q_*$.
\end{Claim}

\begin{Cor}
    The right adjoint to $\Ind$ is $\Res_R = q_*p^!$.
\end{Cor}

\begin{Claim}
    The map $q_*$ sends $\catCoh{\B P}$ to $\catCoh{\B L}$.
\end{Claim}

Hence we can define a left adjoint $q_!$ to $q^!$ as the ind-extension of $\mathbb D_{\mathrm{Serre}} q_* \mathbb \mathbb D_{\mathrm{Serre}}\colon \catCoh{\B P} → \catCoh{\B L}$ (see \cite[Corollary~9.5.9(a)]{Gaitsgory:preprint:IndcoherentSheaves} and \cite{Gaitsgory:preprint:GL.DGcat}).

\begin{Cor}
    The left adjoint to $\Ind$ is $\Res_L = q_!p^*$.
\end{Cor}

Let $P^-$ be the opposite parabolic and write
\[
    \B G \xleftarrow{p^-} \B P^- \xrightarrow{q^-} \B L
\]
and $\Res_L^- = q^-_!(p^-)^*$.

\begin{Claim}[Second Adjointness Theorem]
    \[
        \Res_R \cong \Res_L^-.
    \]
\end{Claim}

This should follow from some version of Braden's \enquote{Hyperbolic Localization} Theorem \cite{Braden:2003:HyperbolicLocalizationOfIC, DrinfeldGaitsgory:arXiv:OnATheoremOfBraden} for quasi- (or ind-) coherent sheaves and the following claim.
\begin{Claim}
    \[\B L \cong \B P ×_{\B G} \B P^-.\]
\end{Claim}
This gives a Cartesian diagram
\[
    \begin{tikzpicture}
        \matrix[commutative diagram] (m) {
            \B L & \B P^- \\
            \B P & \B G \\
        };

        \path[commutative diagram arrows] 
            (m-1-1) edge node[above] {$f^-$} (m-1-2)
                    edge node[left] {$f$} (m-2-1)
            (m-1-2) edge node[right] {$p^-$} (m-2-2)
            (m-2-1) edge node[above] {$p$} (m-2-2);
    \end{tikzpicture}
\]
with $ℂ^*$-actions similar to the ones in \cite{DrinfeldGaitsgory:preprint:GeometricConstantTermFunctors}.
Thus, by the (hypothetical) version of Braden's Theorem and base change, we get
\[
    q_*p^! \cong
    f^*p^! \cong
    {f^-}^! {p^-}^* \cong
    q^-_! {p^-}^*.
\]

\printbibliography
\end{document}
