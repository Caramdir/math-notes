%type: notes
%title: Lie algebroids
%tags: Lie algeroids
\documentclass[english,no-theorem-numbers]{short-notes}

\usepackage{math-alg,math-ag}

\addbibresource{math.bib}
%\bibliography{global.bib}


\title{Lie algebroids}
\author{Clemens Koppensteiner}

\begin{document}

\maketitle

Let $X$ be a complex variety, $\O_X$ its structure sheaf, $\sheaf T_X$ its tangent sheaf and $Ω¹_X$ the sheaf of $1$-forms on $X$ (i.e.~the contangent sheaf).

\begin{Def}
    A \emph{Lie algebroid} consists of 
    \begin{itemize}
        \item an $\O_X$-module $\sheaf A$;
        \item the structure of a $ℂ$-Lie algebra on $\sheaf A$, i.e.\ a skew-symmetric $ℂ$-bilinear pairing $[\,{,}\,]\colon \sheaf A × \sheaf A → \sheaf A$ satisfying the Jacobi identity; and
        \item an $\O_X$-linear map of $ℂ$-Lie algebras $σ\colon A → \sheaf T_X$, called the \emph{anchor map};
    \end{itemize}
    such that the Leibniz identity 
    \[
        [a,fb] = σ(a)(f)b + f[a,b]
    \]
    holds.
\end{Def}

\begin{Exs}\leavevmode
    \begin{itemize}
        \item $\id\colon \sheaf T_X → \sheaf T_X$.
        \item Any sub-$\O$-module of $\sheaf T_X$ with the inclusion map.
            In particular the inclusion of Hamiltonian vector fields on a symplectic variety give a Lie algebroid.
        \item If $X$ is a Poisson manifold, then the Poisson bivector and the Schouten(?) bracket give a Lie algebroid $Ω_X¹ → \sheaf T_X$.
            \qedhere
    \end{itemize}
\end{Exs}

\begin{Q}
    If $X$ is singular, should $\sheaf T_X$ be replaced by the tangent complex $T^\cx X$?
\end{Q}

%\printbibliography
\end{document}
