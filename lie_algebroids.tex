%type: notes
%title: Lie algebroids
%tags: Lie algeroids
\documentclass[english,no-theorem-numbers]{short-notes}

\usepackage{math-alg,math-ag}

\addbibresource{math.bib}
%\bibliography{global.bib}


\title{Lie algebroids}
\author{Clemens Koppensteiner}

\newcommand\at{\operatorname{at}}
\renewcommand\dual{*}

\begin{document}

\newcommand\unit{\operatorname{unit}}
\newcommand\grpd{\mathcal}
\newcommand\algd{\sheaf}

\maketitle

Let $X$ be a complex variety, $\O_X$ its structure sheaf, $\sheaf T_X$ its tangent sheaf and $Ω¹_X$ the sheaf of $1$-forms on $X$ (i.e.~the contangent sheaf).

\begin{Def}
    A \emph{Lie algebroid} consists of 
    \begin{itemize}
        \item an $\O_X$-module $\sheaf A$;
        \item the structure of a $ℂ$-Lie algebra on $\sheaf A$, i.e.\ a skew-symmetric $ℂ$-bilinear pairing $[\,{,}\,]\colon \sheaf A × \sheaf A → \sheaf A$ satisfying the Jacobi identity; and
        \item an $\O_X$-linear map of $ℂ$-Lie algebras $σ\colon A → \sheaf T_X$, called the \emph{anchor map};
    \end{itemize}
    such that the Leibniz identity 
    \[
        [a,fb] = σ(a)(f)b + f[a,b]
    \]
    holds.
\end{Def}

\begin{Exs}\leavevmode
    \begin{itemize}
        \item $\id\colon \sheaf T_X → \sheaf T_X$.
        \item Any sub-$\O$-module of $\sheaf T_X$ with the inclusion map.
            In particular the inclusion of Hamiltonian vector fields on a symplectic variety give a Lie algebroid.
        \item If $X$ is a Poisson manifold, then the Poisson bivector and the Schouten(?) bracket give a Lie algebroid $Ω_X¹ → \sheaf T_X$.
            \qedhere
    \end{itemize}
\end{Exs}

\begin{Q}
    If $X$ is singular, should $\sheaf T_X$ be replaced by the tangent complex $T^\cx X$?
\end{Q}

\section{The Lie algebroid of a Lie groupoid}

Let $\grpd G$ be a Lie groupoid over $X$ with source and target maps $s,t\colon \grpd G → X$ and unit $\unit\colon X → \grpd G$.
To $\grpd G$ we can associate a Lie algebroid $\algd L_{\grpd G}$ in the following way \cite[Definition~3.5.8]{Mackenzie:2005:GeneralTheoryOfLieGroupoidsAndLieAlgebroids}:
The vector bundle underlying $\algd L_{\grpd G}$ is
\[
    \algd L_{\grpd G} = \unit^*\sheaf T_{\grpd G/X},
\]
where $T_{\grpd G/X}$ is the relative tangent bundle of $s\colon \grpd G → X$.
The bracket is given via right-invariant vector fields and the anchor is the defined via the tangent map of $t$.

Since $X \xrightarrow{\unit} \grpd G \xrightarrow{s} X$ is the identity, the triangle
\[
    \unit^* T^*(\grpd G/X) → T^*(X/X) → T^*(X/\grpd G)
\]
of cotangent complexes and hence an isomorphism
\[
    T^*(X/\grpd G) \cong \unit^*(T^*(\grpd G/X))[1] \cong (\algd L_{\grpd G}[-1])^\dual
\]
in the derived category.

\section{Atiyah classes}

Set $J¹\sheaf A = (\O \oplus \sheaf A)^\dual$ (this is an ad-hoc definition; see \cite[Section~4.2.5]{CalaqueVanDenBergh:2010:HochschildCohomologyAndAtiyahClasses}).
There are two monomorphisms $α_i\colon \O → J¹\sheaf A$:
\begin{align*}
    α₁\colon \O → J¹\sheaf A:& f \mapsto \left( (g,a) \mapsto fg + a(1) \right), \\
    α₂\colon \O → J¹\sheaf A:& f \mapsto \left( (g,a) \mapsto gf + a(f) \right).
\end{align*}
Let $\sheaf E$ be an arbitrary $\O$-module.
There exists a short exact sequence of $\O$-modules 
\[
    0 → \sheaf A^\dual \otimes_{\O} \sheaf E → J¹\sheaf A \otimes_{α₂,\O} \sheaf E → \sheaf E → 0,
\]
where the tensor product in the middle term is taken with respect to $α₂$, but the result is viewed as an $\O$-modules via $α₁$.
This sequence corresponds to a class
\[
    \at_{\sheaf A}(\sheaf E) ∈ \Ext¹_\O(\sheaf E,\, \sheaf A^\dual \otimes \sheaf E).
\]
See \cite[Section~8]{CalaqueVanDenBergh:2010:HochschildCohomologyAndAtiyahClasses} and \cite{ChenStienonXu:arXiv:FromAtiyahClassesToHomotopyLeibnizAlgebras} for details.
We also write $\at_{\sheaf A}(\sheaf E)$ for the corresponding morphism $\sheaf E → \sheaf A^\dual \otimes \sheaf E[1]$ in the derived category $D(X)$.
I should check that for $\sheaf A = \sheaf T_X$ this yields the usual Atiyah class \cite[Section~1.1]{Markarian:2009:AtiyahClassHochschildCohomologyRiemannRoch}.

Setting $\sheaf E = \sheaf A^\dual$ we obtain a morphism $\at_{\sheaf A}(\sheaf A^\dual)\colon \sheaf A^\dual → \sheaf A^\dual \otimes \sheaf A^\dual[1]$.
Dually this gives a morphism $\sheaf A \otimes \sheaf A [-1] → \sheaf A$ or 
\[
    \sheaf A[-1] \otimes \sheaf A[-1] → \sheaf A[-1],
\]
i.e.\ a bilinear map on $\sheaf A[-1]$.

\begin{Thm}[{\cite{ChenStienonXu:arXiv:FromAtiyahClassesToHomotopyLeibnizAlgebras}}]
    This map gives $\sheaf A[-1]$ the structure of a Lie algebra object in $D(X)$.
\end{Thm}

It seems likely that one can translate the proof of \cite[Proposition~1]{Markarian:2009:AtiyahClassHochschildCohomologyRiemannRoch}.

\begin{Conjecture}
    If $\sheaf A$ comes from a groupoid, then this Lie algebra structure coincides with the one constructed in \cite[Section~G.1]{ArinkinGaitsgory:arXiv:v2:SingularSupport}.
\end{Conjecture}

\printbibliography
\end{document}
