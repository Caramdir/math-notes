%type: notes
%title: Spectral Eisenstein series functors
%tags: representation theory, geometric Langlands, local systems
\documentclass[english, no-theorem-numbers]{short-notes}

\usepackage{math-alg,math-ag, math-gl}

\addbibresource{global.bib}
%\bibliography{global.bib}


\title{spectral eisenstein series functors}
\author{Clemens Koppensteiner}

\begin{document}

\newcommand\ST[2]{\operatorname{ST}(#1,#2)}

\maketitle

The following is mainly taken from \cite[Section~12]{ArinkinGaitsgory:arXiv:v2:SingularSupport}.

Let $C$ be a curve over a $ℂ$, $G$ a reductive group.
We are interested in the dg category $\catIndCoh{\Loc_G}$ of ind-coherent sheaves on the derived moduli stack of local $G$-systems on $C$.

Let $P$ be a parabolic subgroup of $G$ and $L$ the corresponding Levi.
We write
\[
    p^P\colon \Loc_P → \Loc_G, \qquad q^P\colon \Loc_P → \Loc_L.
\]
The map $p^P$ is schematic and proper and the map $q^P$ is quasi-smooth \cite[Lemma~12.2.2]{ArinkinGaitsgory:arXiv:v2:SingularSupport}.

\begin{Def}
    The \emph{spectral Eisenstein series functor} for $P$ is
    \[
        \Eis^P\colon (p^P)_*^\IndCoh ∘ (q^P)^!\colon \catIndCoh{\Loc_M} → \catIndCoh{\Loc_G}.
    \]
\end{Def}

The functor $(p^P)_*^\IndCoh$ is defined in \cite[Proposition~3.1.1]{Gaitsgory:preprint:IndcoherentSheaves} and the functor $(q^P)^!$ exists by \cite[Theorem~5.2.2]{Gaitsgory:preprint:IndcoherentSheaves}.

For ease of notation, I will sometimes drop the $P$ from the notation.

\begin{Q}
    We are interested in the following questions:
    \begin{itemize}
        \item 
            Let $P$ and $Q$ be two parabolics with equivalents Levis.
            Are the functors $\Eis^P$ and $\Eis^Q$ isomorphic?
            (Compare Proposition~3.1.9 in Sam's thesis.)
        \item 
            Are the images of the $\Eis$ for non-equivalent Levis orthogonal?
    \end{itemize}
\end{Q}

\section*{adjoints}

\begin{Q}
    Does the functor $\Eis$ have adjoints?
\end{Q}

\begin{Rem}\leavevmode
    \begin{itemize}
        \item $p_*^\IndCoh$ has the right adjoint $p^!$ \cite[3.3.7]{Gaitsgory:preprint:IndcoherentSheaves}.
        \item There is an adjunction $(q^{\IndCoh,*},q_*^\IndCoh)$ \cite[Corollary~1.2.5]{ArinkinGaitsgory:arXiv:v2:SingularSupport}.
        \item For general maps, $q^!$ does not need to be adjoint to $q_*^\IndCoh$.
        \item Possibly \cite[Proposition~5.4.2]{Gaitsgory:preprint:IndcoherentSheaves} (compatibility with adjunction for proper maps) can help.
    \end{itemize}
\end{Rem}

\begin{Prop}[{\cite[Corollary~12.2.3]{ArinkinGaitsgory:arXiv:v2:SingularSupport}}]
    The functor $\Eis^P$ sends $\catCoh{\Loc_M}$ to $\catCoh{\Loc_G}$.
\end{Prop}

\begin{Cor}
    $\Eis^P$ has a (continuous) right adjoint.
    We will denote it by $\CT^P$ (for \emph{spectral constant term functor}).
\end{Cor}

\begin{proof}
    The functor $\Eis$ is continuous, so by the adjoint functor theorem \cite[Corollay~5.5.2.9]{Lurie:2009:HigherToposTheory} it has a right adjoint.
    By \cite[Proposition~5.5.7.2]{Lurie:2009:HigherToposTheory}, the right adjoint is continuous.
\end{proof}

\begin{Q}
    Is $\CT$ also a left adjoint?
\end{Q}

\begin{Q}
    How does $\CT$ differ from $q_*^\IndCoh ∘ p^!$?
\end{Q}

This question might be reduced with the following steps:
\begin{itemize}
    \item How does $q_*^\IndCoh$ differ from the right adjoint of $q^!$ (which must exist for the same reasons as $\CT$)?
    \item How does $q^!$ differ from $q^{\IndCoh,*}$ (the left adjoint to $q_*^\IndCoh$)?
    \item Since $q^! = q^{\mathrm{QCoh},!}(\O_{\Loc_L}) \otimes q^{\IndCoh,*}$: what is $q^{\mathrm{QCoh},!}(\O_{\Loc_L})$?
    \item Maybe making the proof of \cite[Lemma~12.2.2]{ArinkinGaitsgory:arXiv:v2:SingularSupport} explicit leads to nicer properties of $q$? See also \cite[Proposition~9.4.5]{ArinkinGaitsgory:arXiv:v2:SingularSupport}.
\end{itemize}

Let $\sheaf K_P^\dual$ denote the dual of $(q^P)^{\mathrm{QCoh},!}(\O_{\Loc_L})$ (in $\catQCoh P$).

\begin{Lem}
    The functor $\CT_P$ is equal to
    \[
        \sheaf F \mapsto q_*^\IndCoh(\sheaf K_P^\dual \otimes p^!(\sheaf F)).
    \]
\end{Lem}

\begin{Rem}
    Assume there exists $\sheaf L ∈ \catIndCoh{\Loc_P}$ such that $\sheaf K = \sheaf L \otimes \sheaf L$ (such $\sheaf L$ does not need to exist in general; for example if $\sheaf K$ is in odd degree).
    Then setting $\Eis(\sheaf F) = p_*^\IndCoh(\sheaf L^\dual \otimes q^!(\sheaf F))$ yields the much more symmetric $\CT(\sheaf G) = q_*^\IndCoh(\sheaf L^\dual \otimes p^!(\sheaf G))$.
    It would also ensure that $\Eis ∘ {\mathbb D} = \mathbb D ∘ \Eis$, at least on $\cat{Coh}$ (see~\cite[Corollary~9.5.9]{Gaitsgory:preprint:IndcoherentSheaves}).

    Of course, it would be even better if $\sheaf K = \sheaf L = \O_{\Loc_P}$.
\end{Rem}

\section*{the \enquote{steinberg functor}}

For two parabolics $P$ and $Q$ with Levis $L$ and $M$, let
\[ 
    \ST P Q =
    \Loc_P ×_{\Loc G} \Loc Q
\]
and denote the two projection functors by $π^P$ and $π^Q$.
Let $α^P = q^P ∘ π^P$ and analogously $α^Q$.

\begin{Cor}
    \[
        \CT^Q ∘ \Eis^P ({\cdot}) = 
        (α^Q)_*^\IndCoh \bigl( (π^Q)^*(\sheaf K_Q^\dual) \otimes (α^P)^!({\cdot})\bigr)\colon
        \catIndCoh{\Loc_L} → \catIndCoh{\Loc_M}.
    \]
\end{Cor}

\begin{proof}
    \begin{align*}
        \CT^Q ∘ \Eis^P (\sheaf F) &=
        (q^Q)_*^\IndCoh\biggl( \sheaf K_Q^\dual \otimes \bigl( (p^Q)^! ∘ (p^P)_*^\IndCoh ∘ (q^P)^!(\sheaf F)\bigr)\biggr) \\ & =
        (q^Q)_*^\IndCoh\biggl( \sheaf K_Q^\dual \otimes \bigl( (π^Q)_*^\IndCoh ∘ (π^P)^! ∘ (q^P)^!(\sheaf F)\bigr)\biggr) \\ & =
        (α^Q)_*^\IndCoh \bigl( (π^Q)^*(\sheaf K_Q^\dual) \otimes (α^P)^!(\sheaf F)\bigr),
    \end{align*}
    where the first equality is (proper) base change and the second one the compatibility of the pushforward with the action of $\cat{QCoh}$.
\end{proof}

\begin{Q}
    Is
    \[
        \sheaf K_Q^\dual =
        \mathbb D_{\Loc_Q}^{\mathrm{naive}} (q^Q)^{\mathrm{QCoh},!}(\O_{\Loc_M}) =
        (q^Q)^* (\mathbb D_{\Loc_M}^{\mathrm{naive}}\O_{\Loc_M}),
    \]
    and hence
    \[
        (π^Q)^*(\sheaf K_Q^\dual) =
        (α^Q)^*(\mathbb D_{\Loc_M}^{\mathrm{naive}}\O_{\Loc_M})?
    \]
    Further, is $\mathbb D_{\Loc_M}^{\mathrm{naive}}\O_{\Loc_M} = \O_{\Loc_M}$?
\end{Q}

\begin{Q}
    Does $\ST P Q$ have a natural (equidimensional) stratification?
\end{Q}

\printbibliography
\end{document}
