%type: notes
%title: Spectral Eisenstein series functors
%tags: representation theory, geometric Langlands, local systems
\documentclass[english]{short-notes}

\usepackage{math-alg,math-ag, math-gl}

\addbibresource{global.bib}
%\bibliography{global.bib}


\title{spectral eisenstein series functors}
\author{Clemens Koppensteiner}

\begin{document}

\newcommand\sym{\mathrm{sym}}
\newcommand\STs[2]{\operatorname{ST}(#2,#1)}
\newcommand\STf[2]{\operatorname{ST}_{#2,#1}}

\maketitle

The following is mainly taken from \cite[Section~12]{ArinkinGaitsgory:arXiv:v2:SingularSupport}.

Let $C$ be a curve over a $ℂ$, $G$ a reductive group.
We are interested in the dg category $\catIndCoh{\Loc_G}$ of ind-coherent sheaves on the derived moduli stack of local $G$-systems on $C$.

Let $P$ be a parabolic subgroup of $G$ and $L$ the corresponding Levi.
We write
\[
    p_P\colon \Loc_P → \Loc_G, \qquad q_P\colon \Loc_P → \Loc_L.
\]
The map $p_P$ is schematic and proper and the map $q_P$ is quasi-smooth \cite[Lemma~12.2.2]{ArinkinGaitsgory:arXiv:v2:SingularSupport}.

\begin{Q}
    Does $q_P$ have more properties than just being quasi-smooth?
    What is its relative dualizing sheaf?
    
    Maybe making the proof of \cite[Lemma~12.2.2]{ArinkinGaitsgory:arXiv:v2:SingularSupport} explicit leads to nicer properties of $q$?
    See also \cite[Proposition~9.4.5]{ArinkinGaitsgory:arXiv:v2:SingularSupport}.
\end{Q}

\begin{Def}
    The \emph{spectral Eisenstein series functor} for $P$ is
    \[
        \Eis_P^G = (p_P)_*^\IndCoh ∘ q_P^!\colon \catIndCoh{\Loc_M} → \catIndCoh{\Loc_G}.
    \]
\end{Def}

The functor $(p_P)_*^\IndCoh$ is defined in~\cite[Proposition~3.1.1]{Gaitsgory:preprint:IndcoherentSheaves} and the functor $q_P^!$ exists by \cite[Theorem~5.2.2]{Gaitsgory:preprint:IndcoherentSheaves}.

When the group $G$ is clear from context, we set $\Eis_P = \Eis_P^G$.
If there is only on parabolic $P$, then $p = p_P$, etc.

\begin{Q}
    We are interested in the following questions:
    \begin{itemize}
        \item 
            Let $P$ and $Q$ be two parabolics with equivalents Levis.
            Are the functors $\Eis_P$ and $\Eis_Q$ isomorphic?
            (Compare Proposition~3.1.9 in Sam's thesis.)
        \item 
            Are the images of the $\Eis$ for non-equivalent Levis orthogonal?
    \end{itemize}
\end{Q}

\section{adjoints}

\begin{Q}
    Does the functor $\Eis$ have adjoints?
\end{Q}

\begin{Rem}\leavevmode
    \begin{itemize}
        \item $p_*^\IndCoh$ has the right adjoint $p^!$ \cite[3.3.7]{Gaitsgory:preprint:IndcoherentSheaves}.
        \item There is an adjunction $(q^{\IndCoh,*},q_*^\IndCoh)$ \cite[Corollary~1.2.5]{ArinkinGaitsgory:arXiv:v2:SingularSupport}.
        \item For general maps, $q^!$ does not need to be adjoint to $q_*^\IndCoh$.
    \end{itemize}
\end{Rem}

\begin{Prop}[{\cite[Corollary~12.2.3]{ArinkinGaitsgory:arXiv:v2:SingularSupport}}]
    The functor $\Eis_P$ sends $\catCoh{\Loc_M}$ to $\catCoh{\Loc_G}$.
\end{Prop}

\begin{Cor}
    $\Eis_P^G$ has a (continuous) right adjoint.
    We will denote it by $\CT_P^G = \CT_P$ (for \emph{spectral constant term functor}).
\end{Cor}

\begin{proof}
    The functor $\Eis$ is continuous, so by the adjoint functor theorem \cite[Corollay~5.5.2.9]{Lurie:2009:HigherToposTheory} it has a right adjoint.
    By \cite[Proposition~5.5.7.2]{Lurie:2009:HigherToposTheory}, the right adjoint is continuous.
\end{proof}

\begin{Q}
    Is $\CT$ also a left adjoint?
\end{Q}

Let $\sheaf K_P^\dual$ denote the dual of $q_P^{\mathrm{QCoh},!}(\O_{\Loc_L})$ (in $\catQCoh{\Loc_P}$).

\begin{Lem}
    The functor $\CT_P$ is equal to
    \[
        \sheaf F \mapsto q_*^\IndCoh(\sheaf K_P^\dual \otimes p^!(\sheaf F)).
    \]
\end{Lem}

\begin{Rem}%
    \label{rem:symmetric}%
    Assume there exists $\sheaf L ∈ \catIndCoh{\Loc_P}$ such that $\sheaf K = \sheaf L \otimes \sheaf L$ (such $\sheaf L$ does not need to exist in general; for example if $\sheaf K$ is in odd degree).
    Then setting $\Eis(\sheaf F) = p_*^\IndCoh(\sheaf L^\dual \otimes q^!(\sheaf F))$ yields the much more symmetric $\CT(\sheaf G) = q_*^\IndCoh(\sheaf L^\dual \otimes p^!(\sheaf G))$.
    It would also ensure that $\Eis ∘ {\mathbb D} = \mathbb D ∘ \Eis$, at least on $\cat{Coh}$ (see~\cite[Corollary~9.5.9]{Gaitsgory:preprint:IndcoherentSheaves}).

    Of course, it would be even better if $\sheaf K = \sheaf L = \O_{\Loc_P}$.
\end{Rem}

\begin{Notation}%
    There are several different ways to define the Eisenstein series functor (and consequently the constant term functor).
    We will use the following notation:
    \begin{itemize}
        \item $\Eis = \Eis^! = p_*^\IndCoh ∘ q^!$ with $\CT = \CT^! = q_*^\IndCoh(\sheaf K_P^\dual \otimes p^!({-}))$.
            This variant is used in \cite{ArinkinGaitsgory:arXiv:v2:SingularSupport}.
        \item $\Eis^* = p_*^\IndCoh ∘ q^*$ with $\CT^* = q_*^\IndCoh ∘ p^!$.
            This variant is used in \cite{Gaitsgory:preprint:OutlineOfGLCForGL2} and denoted simply $\Eis_{\mathrm{spec}}$ there.
        \item $\Eis^\sym$ and $\CT^\sym$ for the versions defined in Remark~\ref{rem:symmetric} (if existing).
    \end{itemize}
    Analogous notation is applied for functors derived from these.
\end{Notation}

\section{the \enquote{steinberg functor}}

In this section let $P$ and $Q$ be two parabolic subgroups of $G$ with Levis $L$ and $M$ respectively.

\begin{Def}
    The \emph{spectral Steinberg functor} is
    \[
        \STf P Q = \CT_Q ∘ \Eis_P \colon \catIndCoh{\Loc_L} → \catIndCoh{\Loc_M}.
    \]
    The \emph{spectral Steinberg stack} is
    \[ 
        \STs P Q =
        \Loc_P ×_{\Loc G} \Loc_Q.
    \]
    The two projections from $\STs P Q$ will be denoted by $π_P$ and $π_Q$.
    Further, let $α_P = q_P ∘ π_P$ and analogously $α_Q$.
\end{Def}

\begin{Cor}
    \[
        \STf P Q ({-}) = 
        (α_Q)_*^\IndCoh \bigl( π_Q^*(\sheaf K_Q^\dual) \otimes α_P^!({-})\bigr).
    \]
\end{Cor}

\begin{proof}
    \begin{align*}
        \CT_Q ∘ \Eis_P (\sheaf F) &=
        (q_Q)_*^\IndCoh\biggl( \sheaf K_Q^\dual \otimes \bigl( p_Q^! ∘ (p_P)_*^\IndCoh ∘ q_P^!(\sheaf F)\bigr)\biggr) \\ & =
        (q_Q)_*^\IndCoh\biggl( \sheaf K_Q^\dual \otimes \bigl( (π_Q)_*^\IndCoh ∘ π_P^! ∘ q_P^!(\sheaf F)\bigr)\biggr) \\ & =
        (α_Q)_*^\IndCoh \bigl( π_Q^*(\sheaf K_Q^\dual) \otimes α_P^!(\sheaf F)\bigr),
    \end{align*}
    where the first equality is (proper) base change and the second one the compatibility of the pushforward with the action of $\cat{QCoh}$.
\end{proof}

\begin{Rem}
    In the symmetric situation of Remark~\ref{rem:symmetric}, one gets
    \[
        \STf P Q ^\sym({-}) =
        \CT^\sym_Q ∘ \Eis^\sym_P ({-}) = 
        (α_Q)_*^\IndCoh \biggl( \bigl( π_Q^*(\sheaf L_Q^\dual) \otimes π_P^*(\sheaf L_P^\dual) \bigr) \otimes α_P^!({-})\biggr),
    \]
    while
    \[
        \STf P Q ^*({-}) =
        \CT^*_Q ∘ \Eis^*_P ({-}) = 
        (α_Q)_*^\IndCoh \bigl( π_P^*(\sheaf K_P) \otimes α_P^!({-})\bigr).
    \]
\end{Rem}

\begin{Q}
    Is
    \[
        \sheaf K_Q^\dual =
        \mathbb D_{\Loc_Q}^{\mathrm{naive}} q_Q^{\mathrm{QCoh},!}(\O_{\Loc_M}) =
        q_Q^* (\mathbb D_{\Loc_M}^{\mathrm{naive}}\O_{\Loc_M}),
    \]
    and hence
    \[
        π_Q^*(\sheaf K_Q^\dual) =
        α_Q^*(\mathbb D_{\Loc_M}^{\mathrm{naive}}\O_{\Loc_M})?
    \]
    Further, is $\mathbb D_{\Loc_M}^{\mathrm{naive}}\O_{\Loc_M} = \O_{\Loc_M}$?
\end{Q}

\begin{Q}
    Does $\STs P Q$ have a natural (equidimensional) stratification?
\end{Q}

\subsection{a \texorpdfstring{$G$}{G}-equivariant formulation}

Let $[P] \cong G/P$ denote the partial flag variety corresponding to the parabolic $P$, i.e.~the set of all conjugates of $P$.

We consider the \enquote{Betti version} of the moduli of local systems.
Let $\Loc_G'$ be the non-stacky moduli.
If $X$ is an elliptic curve this is the (derived) fiber product $\{1\} ×_G G²$ (with $[{,}]\colon G² → G$).
By definition, $\Loc_G = \rquot{\Loc_G'}G$.

\begin{Def}
    Let $\Loc_{[P]}'$ be the dg scheme over $[P]$ with fibers $\Loc_{P'}'$ over $P' ∈ [P]$.
\end{Def}

\begin{Q}
    How to define/construct this in a formally correct way?
\end{Q}

\begin{Claim}
    There is a canonical isomorphism of derived stacks $\Loc_P \cong \rquot{\smash{\Loc_{[P]}'}}G$.
\end{Claim}

We set 
\[
    \Loc_{[P∩Q]}' = \Loc_{[P]}' ×_{\Loc_G'} \Loc_{[Q]}'.
\]
Informally, this is the dg scheme over $[P] × [Q]$ with fibers $\Loc_{P'∩Q'}$ over $(P',Q') ∈ [P]×[Q]$.

\begin{Claim}
    There is a canonical isomorphism of derived stacks $\STs P Q \cong \rquot{\smash{\Loc_{[P∩Q]}'}}G$.
\end{Claim}

Note that the $G$-orbits on $[P]×[Q]$ are indexed by ${}_QW_P = \dquot{W_Q}{W}{W_P}$.
Thus orbits on $[P]×[Q]$ induce a stratification of $\Loc_{[P∩Q]}'$ indexed by $_QW_P$, and hence one of $\STs P Q$.
The strata are denoted $\STs P Q _w$

\begin{Conjecture}
    The strata are equidimensional.
\end{Conjecture}

\begin{Q}
    If $({}^wP,Q) ∈ [P]×[Q]$ is a point in the orbit indexed by $w$, is $\STs PQ = \Loc_{^wP∩Q}$?
\end{Q}

\begin{Q}
    Can the stratification of $\STs P Q$ be obtained without the detour through $\Loc'$?
    How to obtain a stratification for the \enquote{de Rham moduli stack}?
\end{Q}

Next we should define functors ${\STf P Q} _w\colon \catIndCoh{\Loc_L} → \catIndCoh{\Loc_M}$ given by restriction of $α$ and $β$ to $\STs P Q _w$.

\begin{Q}
    Do the functors ${\STf P Q}_w$ respect nilpotent singular support?    
\end{Q}

\begin{Q}
    Is $\STf P Q = \bigoplus_w {\STf P Q}_w$?
\end{Q}

\subsection{a general filtration}

A choice of map $\{\mathrm{pt}\} → C$ yields a morphism $\Loc_G(C) → \mathrm BG$ via pullback (and the forgetful map $\Loc_G → \Bun_G$.
Thus there is a morphism
\[
    γ\colon \STs PQ → \mathrm BQ ×_{\mathrm BG} \mathrm BP = {}_QW_P.
\]
\begin{Def}
    For $w ∈ {}_QW_P$ set
    \begin{align*}
        \STs PQ_w & = γ^{-1}(w), \\
        \STs PQ_{\overline  w} &= γ^{-1}(\overline{\{w\}}).
    \end{align*}
\end{Def}
The set $_QW_P$ is partially ordered by $w₁ \le w₂$ if $\overline{\{w₁\}} ⊆ \overline{\{w₂\}}$.
Hence the stack $\STs PQ$ is filtered by the closed subspaces $\STs PQ_{\overline w}$.
This in turn induces a filtration on $\STf PQ$.

\begin{Q}
    What can we say about this filtration?
\end{Q}

\section{outline}
\subsection{the problem}

By \cite[Corollary~12.3.10]{ArinkinGaitsgory:arXiv:v2:SingularSupport}, the essential images of of $\catQCoh{\Loc_L} ⊂ \cat{IndCoh}_{\mathrm{Nilp}}(\Loc_L)$ under the functor $\Eis_P^G$ for all parabolic subgroups $P$ of $G$ (including $P=G$) generate $\cat{IndCoh}_{\mathrm{Nilp}}(\Loc_G)$.

The goal is to make this statement more precise:
\begin{itemize}
    \item Does it suffice to look at the images of the \enquote{cuspidal} part $\catQCoh{\Loc_L^{\mathrm{irred}}}$?
    \item For two equivalent Levis, are the images the same?
    \item Are the images (of the cuspidal parts) orthogonal?
    \item Can the images (of the cuspidal parts) be described as modules of a monad on $\catQCoh{\Loc_L}$?
\end{itemize}

A further question: How important is the nilpotent singular support condition for this decomposition?

\subsection{the functors}

To show the orthogonality of the images, we have to show that $\Hom(\Eis_L \sheaf F,\, \Eis_M \sheaf G) \ne 0$ for cuspidal $\sheaf F$, $\sheaf G$ only if $M$ and $N$ are equivalent Levis.
To do so we show that $\CT_M ∘ \Eis_L = 0$ if $M$ and $N$ are not equivalent.
Here $\CT$ is a two-sided adjoint to $\Eis$.

This raises the question whether a two-sided adjoint even exists.
By general abstract nonsense, $\Eis$ has a right adjoint, which is easy to describe.
The question of a left adjoint (which does not necessarily need to equal the right adjoint for our purposes) is harder.

In any case, we need to study the functor $\STf P Q = \CT_Q ∘ \Eis_P$.

\subsection{a filtration of the steinberg functor}

The functor $\STf P Q $ can be described via pull-push over the derived stack $\STs P Q$.
The latter comes with a map to $_PW_Q$ which induces a filtration on $\STf P Q$.
The corresponding subquotient functors are ${\STf PQ}_w$ and are given by pull-push over the inverse image of $w$ via $\STs PQ → {}_PW_Q$.

\subsection{a description of the subquotient functors}

We describe ${\STf PQ}_w$ as $\Eis ∘ \CT$ over a common Levi of $^wL$ and $M$.
If $L$ and $M$ are not equivalent (and hence the common Levi proper) this functor has to vanish on \emph{cuspidal} sheaves by definition.

\subsection{describing the images}

\printbibliography
\end{document}
