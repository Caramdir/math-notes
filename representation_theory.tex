%type: notes
%title: Basic definitions of representation theory
%tags: representation theory
\documentclass[english, no-theorem-numbers]{short-notes}

\usepackage{math-alg,math-ag}

\addbibresource{global.bib}
%\bibliography{global.bib}


\title{basic definitions of (geometric) representation theory}
\author{Clemens Koppensteiner}

\begin{document}

\maketitle

Let $G$ be an affine algebraic group over an algebraically closed field $k$ of characteristic $0$.

\begin{Def}
    A \emph{representation} of $G$ is an homomorphism $ρ\colon G → \operatorname{GL}(V)$ for some finite dimensional vector space $V$ over $k$.
\end{Def}

\begin{Def}
    The \emph{character} of a representation $ρ$ is the map $\trace ∘ ρ \colon G → k$.
\end{Def}

Characters are constant on adjoint orbits, i.e.~are elements of $\O(G)^G$ (where the meaning of $\O$ depends on the context).

\begin{Def}
    A \emph{torus} is an affine algebraic group which is isomorphic to a product of $\mathbb G_m = \GL 1$.
\end{Def}

\begin{Def}
    An element of $\GL n$ is called \emph{semisimple} if it is diagonalizable.
    An element of $G$ is \emph{semisimple} if its image is so for any embedding of $G$ into a $\GL n$.
\end{Def}

\begin{Def}
    An element of $\GL n$ is called \emph{unipotent} if all its eigenvalues are $1$.
    An element of $G$ is \emph{unipotent} if its image is so for any embedding of $G$ into a $\GL n$.
\end{Def}

The last two definitions are independent of the embedding of $G$ into $\GL n$ (and the choice of $n$).

\begin{Thm}[Jordan decomposition]
    Any element $g∈G$ can be written uniquely as a product $g=su$ with $s$ semisimple, $u$ unipotent and $[s,u] = 1$.
\end{Thm}

\begin{Def}
    The \emph{radical} $R(G)$ of $G$ is the maximal closed, connected, normal, solvable subgroup of $G$.
    If $R(G) = 1$, then $G$ is called \emph{semisimple}.
\end{Def}

\begin{Def}
    The \emph{unipotent radical} $R_u(G)$ of $G$ is the set of unipotent elements of $R(G)$.
    If $R_u(G) = 1$, then $G$ is called \emph{reductive}.
\end{Def}

$R_u(G)$ is a closed connected normal subgroup of $G$.

\begin{Ex}
    The group $\GL n$ is reductive and $R(\GL n)$ consists of the scalar matrices.
    The groups $\SL n$, $\SO n$ (for $n \ge 3$) and $\Sp{2m}$ are semisimple.
\end{Ex}

\begin{Def}
    A subgroup $P$ of $G$ is called \emph{parabolic} if $\rquot GP$ is projective.
    A \emph{Borel subgroup} of $G$ is a minimal parabolic subgroup.
\end{Def}

\begin{Prop}
    A subgroup of $G$ is a Borel if and only if it is a maximal closed, connected solvable subgroup.
    A subgroup of $G$ is parabolic if and only if it contains a Borel subgroup of $G$.
\end{Prop}

\begin{Def}
    Let $P$ be a parabolic subgroup.
    Then $\rquot{P}{R_u(P)}$ is called a \emph{Levi} of $G$.
\end{Def}

The normalizer of a parabolic $P$ is equal to $P$. 
Thus the set $[P]$ of all conjugates to $P$ in $G$ is isomorphic to $\rquot GP$.
Every Borel subgroups is conjugate

\begin{Def}
    Let $P$ be a parabolic subgroup of $G$.
    Then the projective variety $\rquot GP$ is called the \emph{partial flag variety} corresponding to $P$.
    If $P = B$ is a Borel, then $\mathcal{Fl} = \rquot GB$ is called the \emph{flag variety of $G$}.
\end{Def}

\begin{Def}
    Let $T$ be a torus of $G$. 
    Then the \emph{Weyl group of $T$ in $G$} is
    \[
        W(T) = \rquot{N_G(T)}{Z_G(T)}.
    \]
    where $N_G(T)$ is the normalizer and $Z_G(T)$ the centralizer of $T$ in $G$.
    If $H$ is a maximal torus, then $W = W(H)$ is the \emph{Weyl group} of $G$.
\end{Def}

Any two maximal tori of $G$ are conjugate.
Hence the Weyl group of $G$ is defined up to non-canonical isomorphism.
Note that if $H$ is a maximal torus, then $Z_G(H) = H$.
The Weyl group of $G$ is always finite.

If $T$ is a torus, then $W(T)$ has a natural action on $T$ by conjugation.

Each Borel subgroup $B$ contains a unique maximal torus $H$.
Setting $N = R_u(B)$ one gets a decomposition
\[
    B = HN, \qquad H∩N = \{1\}.
\]

\begin{Thm}[Bruhat decomposition]
    Let $B$ be a Borel subgroup of $G$ with corresponding maximal torus $H$ and Weyl group $W$.
    Then
    \[
        G = \coprod_{w ∈ W} BwB.
    \]
\end{Thm}

Note that the double coset $BwB$ does not depend on the choice of representative of $w$ in $N_G(H)$.

\begin{Def}
    The locally closed subvarieties $X_w = BwB$ are called \emph{Schubert cells} and the closures $\overline{X_w}$ are called \emph{Schubert varieties} of $G$.
\end{Def}

%\printbibliography
\end{document}
