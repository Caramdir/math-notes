\documentclass{ck-article}

\title{Duality for D-modules on stacks}
\author{Clemens Koppensteiner}

\usepackage{math-gl}

\newcommand{\BunG}[1][G]{\Bun_{#1}}
\newcommand{\coev}[1][]{\operatorname{coev}_{#1}}
\newcommand{\ev}[1][]{\operatorname{ev}_{#1}}

\DeclareMathOperator{\Ind}{Ind}
\newcommand{\op}{\mathrm{op}}

\newcommand\cVerD[1][]{\mathbb{D}^{\mathrm{Ve}}_{#1}}
\newcommand\VerD[1][]{\mathbf{D}^{\mathrm{Ve}}_{#1}}

\newcommand\Cotrnk[1]{\operatorname{Cotrnk}(#1)}
\newcommand\catDModco[1]{\catDMod{#1}_{\mathrm{co}}}
\newcommand\copf{\mathrm{co},*}

\newcommand\PsId[1][]{\operatorname{Ps-Id}_{#1}}

\newcommand\renpf{\bigblacktriangleup}

\addbibresource{math.bib}

\begin{document}

\maketitle
\begin{abstract}
    We want to understand duality for the category $\catDMod{\BunG}$ and use it to compute the Hochschild homology of this category.
\end{abstract}

We fix an algebraically closed ground field $k$ of characteristic $0$.

\section{Prologue: Duality for compactly generated dg-categories}

References for this section are \cite{Gaitsgory:preprint:GL.DGcat}, \cite[Sections~1 and~5]{Gaitsgory:arXiv:FunctorsGivenByKernelsAdjunctionsAndDuality} and \cite[Section~2]{BenZviNadler:arXiv:NonlinearTraces}.

A dg-category $\cat{C}$ is called \emph{dualizable}, if it is dualizable in $\cat{dgCat_{cont}}$ with respect to the tensor product $\otimes$ of dg-categories [Should add some references for the correct setup].
Thus we have \emph{dual category} $\cat C^\dual$, a \emph{coevaluation} (or \emph{unit}) functor
\[
    \coev[\cat C]\colon \catVect → \cat C \otimes \cat C^\dual
\]
and an \emph{evaluation} (or \emph{counit}) functor
\[
    \ev[\cat C]\colon \cat C \otimes \cat C^\dual → \catVect,
\]
satisfying the usual compatibilities.
The usual construction gives an equivalence
\[
    \cat{Funct_{cont}}(\cat C, \cat C) \cong \cat C^\dual \otimes \cat C.
\]

\begin{Def}
    The \emph{Hochschild homology} or \emph{trace} of a dualizable dg-category $\cat C$ is the endofuctor
    \[
        (\ev[\cat C] ∘ \coev[\cat C])
    \]
    of $\catVect$ (or equivalently its image at the compact generator $k ∈ \catVect$).
\end{Def}

If $\cat C$ is compactly generated by $\cat C^c$, then we can construct $\cat C^\dual$ as the ind-completion
\[
    \cat C^\dual = \Ind\bigl( (\cat C^c)^\op \bigr) \cong \cat{Funct}(\cat C^c, \catVect).
\]
Thus we have an equivalence
\[
    (\cat C^c)^\op \cong (\cat C^\dual)^c, \qquad x \mapsto x^\dual
\]
From this we see that for $x ∈ \cat C^c$ and $ξ ∈ \cat C^\dual$ we have
\begin{equation}
    \label{eq:ev-via-maps}
    \ev[\cat C](x \otimes ξ) \cong \Maps_{\cat C^\dual}(x^\dual, ξ).
\end{equation}

\begin{Rem}
    Gaitsgory often silently identifies $(\cat C^c)^\op$ with $(\cat C^\dual)^c$ and often just writes $x$ for ${x}^\dual$ (for compact objects).
    We will do so too.
\end{Rem}


\section{Duality for D-modules: the QCA case}

Let now $\stack X$ be a QCA stack over $k$ and $\cat C = \catDMod{\stack X}$.
The category $\catDMod{\stack X}$ is compactly generated \cite[Theorem~8.1.1]{DrinfeldGaitsgory:2013:FinitenessQuestions} and we have a Verdier duality equivalence \cite[Section~8.4]{DrinfeldGaitsgory:2013:FinitenessQuestions}
\[
    \cVerD[\stack X]\colon (\catDMod{\stack X}^c)^\op \isoto \catDMod{\stack X}^c,
\]
which extends to and equivalence
\begin{equation}
    \label{eq:qca:verdier}
    \VerD[\stack X]\colon \catDMod{\stack X}^\dual \isoto \catDMod{\stack X}.
\end{equation}
Thus we have an equivalence
\begin{equation}
    \label{eq:qca:kernel-identification}
    \catDMod{\stack X} \otimes \catDMod{\stack X}^\dual
    \cong
    \catDMod{\stack X} \otimes \catDMod{\stack X}
    \cong
    \catDMod{\stack X × \stack X},
\end{equation}
where the first equivalence is given by $\id \otimes \VerD[\stack X]$ and the second one is \cite[Corollary~8.3.4]{DrinfeldGaitsgory:2013:FinitenessQuestions}.
We want to understand the duality datum for $\catDMod{\stack X}$ under this identification, i.e.~we want to compute
\[
    \coev\colon \catVect → \catDMod{\stack X × \stack X}
    \quad\text{and}\quad
    \ev\colon \catDMod{\stack X × \stack X} → \catVect.
\]
We will first compute the functor $\ev$ on compact objects.
Thus let $\sheaf F, \sheaf G ∈ \catDMod{\stack X}^c$.
Reversing the equivalence \eqref{eq:qca:kernel-identification} and using equation \eqref{eq:ev-via-maps}, we have
\[
    \ev(\sheaf F \boxtimes \sheaf G) \cong
    \Maps_{\catDMod{\stack X}^\dual}(\sheaf F, (\cVerD[\stack X])^{-1} \sheaf G) \cong
    \Maps_{\catDMod{\stack X}}(\cVerD[\stack X]\sheaf F, \sheaf G).
\]
Now, by \cite[Lemma~7.3.5]{DrinfeldGaitsgory:2013:FinitenessQuestions} the last is equivalent to
\[
    \ΓdR(\stack X, \sheaf F \shriektensor \sheaf G).
\]
Thus on compact objects,
\[
    \ev({-}) = \ΓdR(\stack X, Δ^!({-})),
\]
where $Δ\colon \stack X → \stack X × \stack X$ is the diagonal.
The functor of taking de Rham global sections is not continuous on $\catDMod{\stack X}$.
Thus we have to replace it by the renormalized version $\ΓrendR$, which is continuous and agrees with $\ΓdR$ on compact objects (cf.~\cite[Section~9.1]{DrinfeldGaitsgory:2013:FinitenessQuestions}).
In summary we get an isomorphism of functors
\[
    \ev({-}) \cong \ΓrendR(\stack X, Δ^!({-}))\colon \catDMod{\stack X × \stack X} → \catVect.
\]
Under the identification \eqref{eq:qca:verdier}, the dual to $\ΓrendR$ is $p^!$ (where $p\colon \stack X → \pt$ is the structure morphism) and the dual to $Δ_*$ is $Δ^!$, as $Δ$ is schematic (cf.~\cite[Section~9.3]{DrinfeldGaitsgory:2013:FinitenessQuestions}).
Thus we get the following description of the duality data for $\catDMod{\stack X}$.

\begin{Prop}[{\cite[Proposition~8.4.6 and Corollary~9.2.15]{DrinfeldGaitsgory:2013:FinitenessQuestions}}]
    \label{prop:qca:ev-and-coev}%
    \[
        \coev = Δ_* ∘ p^!
        \quad\text{and}\quad
        \ev = \ΓrendR(\stack X, {-}) ∘ Δ^!.
    \]
\end{Prop}

\begin{Cor}[{\cite[Proposition~4.2]{BenZviNadler:arXiv:NonlinearTraces}}]
    The Hochschild homology of $\catDMod{\stack X}$ is given by $\ΓrendR(\ls \stack X,\, ω_{\ls \stack X})$, where $\ls \stack X$ is the derived loop space of $\stack X$.
\end{Cor}

\begin{proof}
    We have to compute
    \[
        \ev ∘ \coev = \ΓrendR(\stack X, {-}) ∘ Δ^! ∘ Δ_* ∘ p^!.
    \]
    But since $Δ$ is schematic, we can do base change along
    \[
        \begin{tikzcd}
            \ls\stack X \arrow[r, "π₁"] \arrow[d, "π₂"] & \stack X \arrow[d, "Δ"] \\
            \stack X \arrow[r, "Δ"] & \stack X × \stack X
        \end{tikzcd}
    \]
    and have an identification
    \[
        \ΓrendR(\stack X, {-}) ∘ Δ^! ∘ Δ_* ∘ p^! =
        \ΓrendR(\stack X, {-}) ∘ (π₂)_* ∘ π₁^! ∘ p^! =
        \ΓrendR(\ls\stack X, {-}) ∘ (p ∘ π₁)^!.
        \qedhere
    \]
\end{proof}

\subsection*{Functors given by kernels}

In view of \eqref{eq:qca:verdier}, for QCA stacks $\stack X$ and $\stack Y$ we have an equivalence
\[
    \cat{Funct_{cont}}(\catDMod{\stack X},\, \catDMod{\stack Y}) \cong
    \catDMod{\stack X}^\dual \otimes \catDMod{\stack Y} \cong
    \catDMod{\stack X × \stack Y}.
\]
This construction is very concrete, as the following lemma shows.
\begin{Lem}
    Under this identification, the functor corresponding to the kernel $\sheaf Q ∈ \catDMod{\stack X × \stack Y}$ is given by
    \[
        F_{\sheaf Q}\colon \sheaf F \mapsto (p_{\stack Y})_{\renpf}(\sheaf Q \shriektensor p_{\stack X}^! \sheaf F),
    \]
    where $p_{\stack X}\colon \stack X × \stack Y → \stack X$ and $p_{\stack Y}\colon \stack X × \stack Y → \stack Y$ are the projection morphisms.
\end{Lem}

\begin{proof}
    For two dualizable dg-categories $\cat C$ and $\cat D$, the functor corresponding to a kernel $Q ∈ \cat C^\dual \otimes \cat D$ is given by the composition
    \[
        \cat C
        \xrightarrow{\id[\cat C] \otimes Q}
        \cat C \otimes \cat C^\dual \otimes \cat D
        \xrightarrow{\ev[\cat C] \otimes \id[\cat D]}
        \cat D.
    \]
    By Proposition~\ref{prop:qca:ev-and-coev}, in our case this is the functor
    \[
        \catDMod{\stack X} → \catDMod{\stack X × \stack X × \stack Y} → \catDMod{\stack Y}
    \]
    given by
    \[
        \sheaf F \mapsto \sheaf F \boxtimes \sheaf Q \mapsto
        (p_{\sheaf Y})_{\renpf}(Δ_{\stack X} × \id[\stack Y])^! (\sheaf F \boxtimes \sheaf Q),
    \]
    where $Δ_{\stack X} × \id[\stack Y]\colon \stack X × \stack Y → \stack X × \stack X × \stack Y$ is the map given by $(x, y) \mapsto (x, x, y)$.
    The same map can be described as $(p_{\stack X}, \id[\stack X × \stack Y])$.
    Thus,
    \[
        (Δ_{\stack X} × \id[\stack Y])^! (\sheaf F \boxtimes \sheaf Q) \cong
        (p_{\stack X}, \id[\stack X × \stack Y])^! (\sheaf F \boxtimes \sheaf Q) \cong
        Δ_{\stack X × \stack Y}^!(p_{\stack X}^! \sheaf F \boxtimes \sheaf Q) \cong
        p_{\stack X}^! \sheaf F \shriektensor \sheaf Q.
        \qedhere
    \]
\end{proof}

\begin{Ex}
    \label{ex:qca:tensoring_kernel}%
    Let $\sheaf M ∈ \catDMod{\stack X}$.
    The functor $\sheaf F \mapsto \sheaf M \shriektensor \sheaf F$ is given by the kernel $Δ_* \sheaf M$.
    This follows from the projection formula (note that $Δ$ is schematic):
    \[
        (p₂)_{\renpf}(Δ_*\sheaf M \shriektensor p₁^! \sheaf F) =
        (p₂)_{\renpf}Δ_*(\sheaf M \shriektensor Δ^!p₁^! \sheaf F) =
        (\id[\stack X])_*(\sheaf M \shriektensor \id[\stack X]^! \sheaf F) =
        \sheaf M \shriektensor \id[\stack X]^! \sheaf F. \qedhere
    \]
\end{Ex}

\begin{Lem}
    Let $\stack X$, $\stack Y$ and $\stack Z$ be QCA stacks.
    Let $\sheaf P ∈ \catDMod{\stack X × \stack Y}$ and $\sheaf Q ∈ \catDMod{\stack Y × \stack Z}$.
    Then the functor $F_{\sheaf Q} ∘ F_{\sheaf P}$ is given by the kernel
    \[
        (p_{\stack X, \stack Z})_{\renpf}(p_{\stack X, \stack Y}^! \sheaf P \shriektensor p_{\stack Y, \stack Z}^! \sheaf Q) ∈ \catDMod{\stack X × \stack Z}.
    \]
\end{Lem}

\section{Duality for D-modules: the non-quasicompact case}

From now on the stack $\stack X$ will no longer be assumed to be quasi-compact.
Then we are unfortunately no longer guaranteed that $\catDMod{\stack X}$ is compactly generated.
To work around this problem, Drinfeld and Gaitsgory introduced the notion of truncatability.

\subsection{Truncatable stacks}

The general references for this section is \cite[Section~4]{DrinfeldGaitsgory:arXiv:CompactGenerationOfDModOnBunG}.
There is also a summary in \cite{Gaitsgory:arXiv:FunctorsGivenByKernelsAdjunctionsAndDuality}.

\begin{Def}[{\cite[Sections~3 and~4.1]{DrinfeldGaitsgory:arXiv:CompactGenerationOfDModOnBunG}}]
    An open embedding $j\colon \stack Y₁ \hookrightarrow \stack Y₂$ of QCA stacks is called \emph{co-truncative} if the functor $j^!$ has a (continuous) left adjoint $j_!$, defined on all of $\catDMod{\stack Y₁}$.
    An open substack $\stack U$ of $\stack X$ (where $\stack X$ is not assumed to be quasi-compact) is called \emph{co-truncative} if for any quasi-compact open substack $\stack U'$ of $\stack X$ the inclusion $j\colon \stack U ∩ \stack U' \hookrightarrow \stack U'$ is co-truncative.
    The stack $\stack X$ is called \emph{truncatable} if it can be covered by co-truncative open substacks.
    We write $\Cotrnk{\stack X}$ for the poset of quasi-compact open co-truncative substacks of $\stack X$.
\end{Def}

If $\stack X$ is truncatable, then there is an equivalence \cite[Corollary 4.2.3]{DrinfeldGaitsgory:arXiv:CompactGenerationOfDModOnBunG}
\[
    \catDMod{\stack X} \cong
    \lim_{\stack U ∈ \Cotrnk{\stack X}^{\op}} \catDMod{\stack U}.
\]
The point of all this are the following theorems.

\begin{Thm}[{\cite[Corollary~4.1.6]{DrinfeldGaitsgory:arXiv:CompactGenerationOfDModOnBunG}}]\label{thm:non-qca:compact-generation}%
    If $\stack X$ is truncatable, then $\catDMod{\stack X}$ is compactly generated by the elements of the form
    \[
        j_!\sheaf F, \quad (j\colon \stack U \hookrightarrow \stack X) ∈ \Cotrnk{\stack X},\, \sheaf F ∈ \catDMod{\stack U}^c.
    \]
\end{Thm}

\begin{Thm}[{\cite[Theorem~4.1.8]{DrinfeldGaitsgory:arXiv:CompactGenerationOfDModOnBunG}}]
    The stack $\BunG$ is truncatable and hence $\catDMod{\BunG}$ is compactly generated.
\end{Thm}

From now on we assume that $\stack X$ is truncatable.
If we want to understand duality for $\catDMod{\stack X}$ we need to have a category that is generated by the duals of the objects of Theorem~\ref{thm:non-qca:compact-generation}.
Thinking through the identification and the yoga of dual categories we are led to the following definition.
\begin{Def}
    We define a category
    \[
        \catDModco{\stack X} = \lim_{\Cotrnk{\stack X}^{op}} \cat{DMod^?},
    \]
    where $\cat{DMod}^?\colon \Cotrnk{\stack X} → \cat{dgCat_{cont}}$ is the functor that sends $\stack U$ to $\catDMod{\stack U}$ and $j\colon \stack U₁ \hookrightarrow \stack U₂$ to the right adjoint $j^?$ of $j_*$.
\end{Def}

For $(j\colon \stack U \hookrightarrow \stack X) ∈ \Cotrnk{\stack X}$, we denote by $j^?\colon \catDMod{\stack X} → \catDMod{\stack U}$ the tautological evaluation functor and by $j_{\copf}$ its left adjoint.
The category $\catDModco{\stack X}$ is compactly generated by objects of the form
\[
    j_{\copf}\sheaf F,
    \quad (j\colon \stack U \hookrightarrow \stack X) ∈ \Cotrnk{\stack X},\, \sheaf F ∈ \catDMod{\stack U}^c.
\]
We obtain a Verdier duality functor
\[
    \cVerD[\stack X]\colon (\catDMod{\stack X}^c)^\op → \catDModco{\stack X}^c
\]
by
\[
    \cVerD[\stack X](j_! \sheaf F) = j_{\copf}(\cVerD[\stack U](\sheaf F)),
    \quad (j\colon \stack U \hookrightarrow \stack X) ∈ \Cotrnk{\stack X},\, \sheaf F ∈ \catDMod{\stack U}^c,
\]
which extends to an equivalence
\begin{equation}
    \label{eq:non-qca:verdier}
    \VerD[\stack X]\colon \catDMod{\stack X}^\dual \isoto \catDModco{\stack X}.
\end{equation}

\subsection{The pseudo-identity}

In general the categories $\catDModco{\stack X}$ and $\catDMod{\stack X}$ are different.
However it is easy to define functors $\catDModco{\stack X} → \catDMod{\stack X}$.
For this observe that Verdier duality gives an identification
\begin{equation}
    \label{eq:not-qca:co-kernel-identification}
    \catDModco{\stack X}^\dual \otimes \catDMod{\stack X} \cong
    \catDMod{\stack X} \otimes \catDMod{\stack X} \cong
    \catDMod{\stack X × \stack X},
\end{equation}
where the last equivalence follows from the proof of~\cite[Corollary~8.3.4]{DrinfeldGaitsgory:2013:FinitenessQuestions}.
Thus functors $\catDModco{\stack X} → \catDMod{\stack X}$ correspond to D-modules on $\stack X × \stack X$.
A naive guess for a good kernel would be $Δ_*ω_{\stack X}$, since this is the kernel corresponding to the identity of $\stack X$ was quasi-compact.
However this turns to be the wrong choice \cite[Proposition~4.4.5]{DrinfeldGaitsgory:arXiv:CompactGenerationOfDModOnBunG}.

\begin{Def}[{\cite[Paragraph~7.3.4]{Gaitsgory:arXiv:FunctorsGivenByKernelsAdjunctionsAndDuality}}]
    The \emph{pseudo-identity} is the functor
    \[
        \PsId[\stack X]\colon \catDModco{\stack X} → \catDMod{\stack X}
    \]
    corresponding to the kernel $Δ_!k_{\stack X}$ under the identification \eqref{eq:not-qca:co-kernel-identification}.
\end{Def}

\begin{Def}[{\cite[Paragraph~7.6.3]{Gaitsgory:arXiv:FunctorsGivenByKernelsAdjunctionsAndDuality}}]
    A truncatable stack $\stack X$ is called \emph{miraculous} if $\PsId[\stack X]$ is an equivalence.
\end{Def}

\begin{Thm}[{\cite[Theorem~7.6.5]{Gaitsgory:arXiv:FunctorsGivenByKernelsAdjunctionsAndDuality}}]
    The stack $\BunG$ is miraculous.
\end{Thm}

Thus for a miraculous stack $\stack X$ we have an equivalence
\begin{equation}
    \label{eq:non-qca:miraculous-duality}
    \PsId[\stack X] ∘ \VerD[\stack X] \colon
    \catDMod{\stack X}^\dual \isoto
    \catDModco{\stack X} \isoto
    \catDMod{\stack X}.
\end{equation}

\subsection{Duality data?}

From now on we assume that $\stack X$ is miraculous.
Then \eqref{eq:non-qca:miraculous-duality} induces an equivalence
\begin{equation}
    \begin{aligned}
        \label{eq:non-qca:kernel-identification}
        \catDMod{\stack X} \otimes \catDMod{\stack X}^\dual
        & \cong
        \catDMod{\stack X} \otimes \catDModco{\stack X}
        \\ & \cong
        \catDMod{\stack X} \otimes \catDMod{\stack X}
        \cong
        \catDMod{\stack X × \stack X}.
    \end{aligned}
\end{equation}
We would like to know what the evaluation functor $\catDMod{\stack X} \otimes \catDMod{\stack X}^\dual → \catVect$ looks like under this identification, i.e.~we want to describe the functor
\[
    \ev\colon \catDMod{\stack X × \stack X} → \catVect.
\]
As before, we use \eqref{eq:ev-via-maps} to get
\begin{align*}
    \ev(\sheaf F \boxtimes \sheaf G)
    & \cong
    \Maps_{\catDMod{\stack X}^\dual}\bigl(\sheaf F,\, (\cVerD[\stack X])^{-1}(\PsId[\stack X]^{-1}(\sheaf G))\bigr)
    \\ & \cong
    \Maps_{\catDModco{\stack X}}\bigl(\cVerD[\stack X](\sheaf F),\, \PsId[\stack X]^{-1}(\sheaf G)\bigr)
\end{align*}
for $\sheaf F, \sheaf G ∈ \catDMod{\stack X}^c$.
Using \cite[Lemma~7.2.4]{Gaitsgory:arXiv:FunctorsGivenByKernelsAdjunctionsAndDuality} we then obtain
\[
    \ev(\sheaf F \boxtimes \sheaf G)
    \cong
    p_{\renpf}(\sheaf F \shriektensor \PsId[\stack X]^{-1}(\sheaf G)),
\]
where $p\colon \stack X → \pt$ is the structure morphism.
\begin{Rem}
    The notation here needs some explanation.
    First, $p_\renpf\colon \catDModco{\stack X} → \catVect$ is the dual to $p^!$ under the identification \eqref{eq:non-qca:verdier}.
    Second there is a canonical action of $\catDMod{\stack X}$ on $\catDModco{\stack X}$, which we denote by $\shriektensor$.
    It is defined so that
    \[
        \sheaf F \shriektensor j_{\copf}(\sheaf G_{\stack U}) = j_{\copf}(j^*\sheaf F \shriektensor \sheaf G_{\stack U})
    \]
    for $(j\colon \stack U \hookrightarrow \stack X) ∈ \Cotrnk{\stack X}$ and $\sheaf G_{\stack U} ∈ \catDMod{\stack U}$.
    For both notions we refer to \cite[Section~7.2]{Gaitsgory:arXiv:FunctorsGivenByKernelsAdjunctionsAndDuality}.
\end{Rem}

\begin{Q}
    Is it possible to express $\ev\colon \catDMod{\stack X × \stack X} → \catVect$ purely in terms of the (actual) D-module categories and without reference to the pseudo-identity?
    If not, can we still describe the Hochschild homology (there should be a pseudo-inverse in the coevaluation map, that might cancel out the inverse one in the evaluation map)?
\end{Q}

\begin{Rem}
    It would be great if $\sheaf F \shriektensor \PsId[\stack X]^{-1}(\sheaf G) \cong \PsId[\stack X]^{-1}(\sheaf F \shriektensor \sheaf G)$.
    While this statement is true for separated schemes (where $\PsId$ is given by tensoring with $k_X$), it is not true for miraculous QCA stacks:
    If it was true for a miraculous QCA stack (with non-proper diagonal), than we would also have $\PsId[\stack X](\sheaf F) \shriektensor \sheaf G \cong \PsId[\stack X](\sheaf F \shriektensor \sheaf G)$.
    Setting $\sheaf F = ω_{\stack X}$, this implies that $\PsId[\stack X]$ is given by tensoring with $\PsId[\stack X](ω_{\stack X})$.
    But then by Example~\ref{ex:qca:tensoring_kernel} its kernel would be $Δ_*\PsId[\stack X](ω_{\stack X})$ instead of $Δ_!k_{\stack X}$.
\end{Rem}

\printbibliography

\end{document}
