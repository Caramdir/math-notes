%type: questions
%title: Lie algebroid modules
%tags: Lie algeroids, Lie algebroid modules, D-modules, symplectic varieties
\documentclass[english,no-theorem-numbers]{short-notes}

\usepackage{math-alg,math-ag}

\addbibresource{global.bib}
%\bibliography{global.bib}


\title{Lie algebroid modules}
\author{Clemens Koppensteiner}

\begin{document}

\maketitle

Seeing Lie algebroids as a generalization of the tangent sheaf, one can see Lie algebroid modules (or modules over the universal enveloping algebra) as generalizations of $D$-modules.
In particular I am interested in modules for Hamiltonian functions on a symplectic variety (or more generally a Poisson variety).

Since Lie algebroid modules are the same as modules over the enveloping algebra, their derived category should exist and be decently well-behaved.

Let $σ\colon \sheaf A → \sheaf T_X$ be a Lie algebroid.

\begin{itemize}
    \item For proving stuff on singular varieties it would be useful to have coordinate-free proofs of basic properties of $D$-modules.
    \item If $\sheaf A$ is Hamiltonian vector fields on a singular symplectic variety, how do $\sheaf A$-modules compare to $D$-modules?
    \item Should $\sheaf A$ be coherent? Is the $\O_X$-module generated by Hamiltonian vector fields always coherent?
    \item What corresponds to $Ω_X = \bigwedge^n Ω¹_X$?
        \begin{itemize}
            \item Should one take the complex $\Hom(\bigwedge^n \sheaf A, \O_X)$ (or some version of that)?
            \item Can we define it by a version of~\cite[Lemma~1.5.7]{HottaTakeuchiTanisaki:2008:DModulesPerverseSheavesRepresentationTheory}?
            \item It should have all the necessary properties to switch between right and left modules.
                In particular it needs to be an $\sheaf A$-module.
        \end{itemize}
    \item What are \enquote{integrable connections}, i.e.~coherent $\sheaf A$-modules?
    \item Do we get a version of Kashiwara equivalence?
    \item Is there a good duality functor?
        \begin{itemize}
            \item On \enquote{integrable connections} it should coincide with coherent duality (up to shift).
        \end{itemize}
    \item Is there a good theory of singular support?
        \begin{itemize}
            \item Is there an analogue of involutivity, and does it hold?
            \item Hence, can one define \enquote{holonomic} $\sheaf A$-modules.
            \item (As a side question: Is it possible to see singular support for $D$-modules in a way similar to singular support for coherent sheaves~\cite{ArinkinGaitsgory:arXiv:SingularSupport}? What is the Hochschild cohomology of the category of $D$-modules? [David said that $D$-modules are coherent modules on the loop space and then singular support of coherent sheaves there matches the characteristic variety of $D$-modules.])
        \end{itemize}
    \item Can one translate regularity?
    \item Is there a version of the Riemann-Hilbert correspondence?
        \begin{itemize}
            \item What is the analogue of constructible sheaves?
            \item As a preparation: Compute the de Rham or solution complexes of some \enquote{simple} $\sheaf A$-modules.
        \end{itemize}
    \item How does all of that interact with desingularization (in particular in the case of a symplectic resolution)?
\end{itemize}

\section*{Examples}

\subsection*{\texorpdfstring{$\as 1$}{A¹}}

We consider $X = \as 1 = \Spec ℂ[x]$ and $\sheaf A \subseteq \sheaf T_X$ the sheaf of vector fields that vanish at $0$, i.e.~$\sheaf A = ℂ[x]x∂_x$.
The universal enveloping algebra of $\sheaf A$ is the subalgebra of $D_X$ generated by $\O_X$ and $x∂_x$.
The associated graded is isomorphic to $ℂ[x,y]$.

There are three basic examples of $\sheaf A$-modules:
\begin{enumerate}
    \item $M = \O_X = \rquot{U(\sheaf A)}{U(\sheaf A)x∂_x}$.
        This has singular support $\{y = 0\}$ and 
        \[ \operatorname{Sol}M = ℂ_X \xrightarrow{0} ℂ_0. \]
    \item $M = \rquot{U(\sheaf A)}{U(\sheaf A)x}$ has singular support $\{x = 0\}$ and
        \[ \operatorname{Sol}M = ℂ_0[-1]. \]
        This should not be considered holonomic.
    \item $M = \rquot{U(\sheaf A)}{U(\sheaf A)x + U(\sheaf A)x∂_x} = ℂ_0$ has singular support $\{(0,0)\}$ and 
        \[ \operatorname{Sol}M = 0. \]
        An explicit resolution of this $U(\sheaf A)$-module is given by
        \[
            U(\sheaf A)
            \xrightarrow{
                \begin{psmallmatrix}
                    x∂_x-1 \\
                    -x
                \end{psmallmatrix}
            }
            U(\sheaf A)^{\oplus 2}
            \xrightarrow{
                (x,\, x∂_x)
            }
            U(\sheaf A),
        \]
        following from $[x, x∂_x] = -x$.
\end{enumerate}
To contrast this with $D$-modules on $X$ we note that $\rquot{D}{D∂_x}$ and its solutions $ℂ_X$ do not appear.
On the other hand the $D$-modules $\rquot{D}{Dx + D∂_x}$ and $\rquot{D}{Dx + Dx∂_x}$ vanish (as $∂_xx = 1 - x∂_x$).

\printbibliography
\end{document}
