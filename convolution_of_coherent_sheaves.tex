%type: notes
%title: Convolution of coherent sheaves
%tags: convolution, coherent sheaves, symmetric monoidal categories
\documentclass[english,no-theorem-numbers]{short-notes}

\usepackage{math-alg,math-ag}

\addbibresource{global.bib}
%\bibliography{global.bib}

\title{Convolution of coherent sheaves}
\author{Clemens Koppensteiner}

\let\conv\star

\begin{document}

\maketitle

Let $X$ be a scheme over $Y$. 
Consider $G = X ×_Y X$ (derived fiber product). 
Then for $\sheaf F₁, \sheaf F₂ ∈ \catQCoh G$ we have the convolution product
\[
\sheaf F₁ \conv \sheaf F₂ = {p₁₃}_*\left(p_{12}^*\sheaf F₁ \otimes p₂₃^*\sheaf F₂\right).
\]
(Of course, all operations are derived.)
This gives $\catQCoh G$ the structure of a monoidal category with unit $\O_Δ = Δ_*\O_X$, where where $Δ\colon X → G$ is the diagonal map (follows from the projection formula for $Δ$).
Then we obtain an action of $\catQCoh G$ on $\catQCoh X$ by setting
\[
\sheaf M \cdot \sheaf F = {p₂}_*\left( \sheaf M \otimes p₁^* \sheaf F\right), \qquad \sheaf M ∈ \catQCoh G,\, \sheaf F ∈ \catQCoh X.
\]
By a standard lemma about Fourier-Mukai transforms this is indeed an action, i.e.\ the convolution of two kernels is the kernel for the composition.

From this we get an action of the dg-algebra $\End_{\catQCoh G}(\O_Δ)$ on each object of $\catCoh X ⊆ \catQCoh X$.

As a particular example we get a convolution on the based loop space $\{0\}×_Y\{0\}$.

%\printbibliography
\end{document}
