%type: notes
%title: Monads
%tags: algebraic geometry, category theory
\documentclass[english,no-theorem-numbers]{short-notes}

\usepackage{math-alg,math-ag, math-gl}
\usepackage{xparse}

\addbibresource{global.bib}
%\bibliography{global.bib}

\newcommand\catSpaces{\cat{Spaces}}
%\newcommand\catSheaves[1]{\cat{Sh}(#1)}
\newcommand\incl{\operatorname{incl}}

\title{monads}
\author{Clemens Koppensteiner}

\begin{document}

\maketitle

We collect some standard category-theoretic constructions.
For simplicity, we mostly use the language of 1-categories, but all concepts make sense in the setting of (stable) $∞$-categories or (co-complete) dg-categories.

\section*{monads}

\begin{Def}
    A \emph{monad} on a category $\cat C$ is a triple $(T, η, μ)$ consisting of
    \begin{itemize}
        \item an endofunctor $T\colon \cat C → \cat C$,
        \item a natural transformation $η\colon \id_{\cat C} → T$, and
        \item a natural transformation $μ\colon T∘T → T$
    \end{itemize}
    such that the appropriate \emph{coherence conditions} hold.
\end{Def}

\begin{Def}
    Let $(T,η,μ)$ be a groupoid in $\cat C$.
    A \emph{$T$-module} (or \emph{$T$}-algebra) is an object $A ∈ \cat C$ together with a morphism $ϑ\colon TA → A$ such that the obvious diagrams commute.

    A \emph{morphism} between $T$-modules $(A,ϑ_A)$ and $(B,ϑ_B)$ is a morphism $f\colon A → B$ such that $ϑ_B ∘ Tf = f∘ϑ_A$.

    The category of all $T$-modules in $\cat C$ will be denoted $\catModules T$ or $\cat C^T$.
\end{Def}

\section*{adjunctions}

\begin{Def}
    An \emph{adjunction} is a pair of functors
    \[
        F\colon \cat C \leftrightarrows \cat D \cocolon G
    \]
    together with natural transformations
    \[
        η\colon \id[\cat C] → GF
        \qquad\text{and}\qquad
        ε\colon FG → \id[\cat D],
    \]
    called \emph{unit} and \emph{counit}, such that the compositions
    \[
        F \xrightarrow{Fη} FGF \xrightarrow{εF} F
        \qquad\text{and}\qquad
        G \xrightarrow{ηG} GFG \xrightarrow{Fε} G
    \]
    are the identity transformations.
\end{Def}

This gives a natural isomorphism
\[
    \Hom_{\cat D}(F({-}),{-}) \cong \Hom({-},G({-})).
\]

\begin{Construction}
    Let $F\colon \cat C \rightleftarrows \cat D\cocolon G$ be a pair of adjoint functors.
    Then $T = GF$ is a monad on $\cat C$.
    Explicitly, the unit of $T$ is equal to the unit of the adjunction and the multiplication map is $GεF\colon GFGF → GF$.
\end{Construction}

\begin{Thm}[Barr--Beck--Lurie, \enquote{easy version}]
    Let $F\colon \cat C \rightleftarrows \cat D\cocolon G$ be a pair of adjoint functors between co-complete dg-categories and assume that $G$ commutes with colimits (equivalently, $F$ preserves compactness).
    Let $T = GF$ be the corresponding monad on $\cat C$.
    Then $\cat D \cong \catModules{T}$ if and only if $G$ is conservative.
\end{Thm}

\section*{groupoids}

\textbf{Warning:} There might be some confusion in the use of the words \enquote{source} and \enquote{target}.

We let $\catSpaces$ be an appropriate category of algebraic spaces and $\cat{Sh}({-})$ be an appropriate sheaf theory on $\catSpaces$ (I have ind-coherent sheaves on derived stacks in mind, but we are only going to use some basic functors and adjunctions that hold in a large variety of cases).

\begin{Def}
    A \emph{groupoid} $\mathcal G$ in $\catSpaces$ consists of a space $G₀$ of \enquote{objects} and a space $G₁$ of \enquote{morphisms} with
    \begin{itemize}
        \item \emph{source} and \emph{target} maps $s,t\colon G₁ \leftleftarrows G₀$,
        \item a \emph{unit} $e\colon G₀ → G₁$,
        \item a \emph{multiplication} (or \emph{composition}) map $m\colon G₁ ×\limits_{s,G₀,t} G₁ → G₁$,
        \item a \emph{inverse} map $ι\colon G₁ → G₁$,
    \end{itemize}
    such that
    \begin{itemize}
        \item $s ∘ e = t ∘ e = \id_{G₀}$,
        \item $s ∘ m = s ∘ p₂$ and $t ∘ m = t ∘ p₁$ (where $p_i\colon G₁ ×_{s,G₀,t} G₂$ are the projection maps.
        \item $m$ is associative,
        \item $ι$ interchanges $s$ and $t$ and is an inverse for $m$.
    \end{itemize}
\end{Def}

\begin{Ex}
    The basic example is the following:
    Let $f\colon X → S$ be a map in $\catSpaces$. 
    We set $G_0 = X$ and $G₁ = X ×_S X$.
    The source and target maps are given by $p₁$ and $p₂$, the unit by the diagonal $Δ\colon X → X×_SX$, the inverse by interchanging the factors and multiplication is $p₁₃\colon X ×_S X ×_S X → X×_SX$.
\end{Ex}

\subsection*{monads from groupoids}

Let $\mathcal G$ be a groupoid in $\catSpaces$.
Then we can give $s_*t^!$ the structure of a monad on $\catSheaves{G₀}$ in the following way.

\begin{itemize}
    \item $T = s_*t^!$.
    \item By $(e_*,e^!)$-adjunction (assuming eg.~that $e$ is a closed embedding) we have a transformation 
        \[
            \id = (s∘e)_*(t∘e)^! = s_*e_*e^!t^! → s_*t^!.
        \]
    \item Consider the following commutative diagram
        \[
            \begin{tikzcd}[column sep=small]
                {}& & G₁ \arrow[bend right]{dddll}{s} \arrow[bend left]{dddrr}{t} & & \\
                & & G₁ ×_{G₀} \arrow{u}{m}\arrow{dl}{π₁}\arrow{dr}{π₂} G₁ & & \\
                & G₁ \arrow{dl}{s}\arrow{dr}{t} & & G₁ \arrow{dl}{s}\arrow{dr}{t} & \\
                G₀ & & G₀ & & G₀
            \end{tikzcd}
        \]
        with Cartesian middle square.
        Then base change and $(m_*,m^!)$-adjuction (assuming e.g.~that $m$ is proper) gives a transformation
        \[
            T² =
            s_*t^!s_*t^! =
            (s∘π₁)_*(t∘π₂)^! =
            (s∘m)_*(t∘m)^! =
            s_*m_*m^!t^! →
            s_*t^! =
            T.
        \]
\end{itemize}

\subsubsection*{Restricting to the unit}

Assume that $e$ is a closed embedding and let $E$ be its image in $G₁$.
We have adjoint functors
\[
    \operatorname{incl}\colon \catSheaves{G₁}_E \rightleftarrows \catSheaves{G₁}\cocolon Γ_E.
\]
Explicitly, $Γ_E \sheaf F$ is given as $\operatorname{Cone}\bigl(\sheaf F → j_*j^*\sheaf F\bigr)[-1]$.
Assuming e.g.~that $s$ and $t$ are proper, we get adjoint functors
\[
    (s_E)_* = \incl ∘ s_*\colon \catSheaves{G₀} \rightleftarrows \catSheaves{G₁}_E \cocolon s_E^! = s^! ∘ Γ_E
\]
and analogously for $t$.
As above, the groupoid $\mathcal G$ induces the structure of a monad on $(s_E)_*t_E^!$.

Consider now the case $\mathcal G = X×_SX$ with $f\colon X → S$ proper.
Let $p₁ = t$ and $p₂ = s$ be the two projections and let $(p_{2,Δ})_*$ and $p_{1,Δ}^!$ be defined as above.
\begin{Prop}
    Let $\sheaf F ∈ \catSheaves{S}$.
    Then $f^!\sheaf F$ has a canonical structure as a $(p_{2,Δ})_*p_{1,Δ}^!$-module.
\end{Prop}

\begin{proof}
    By $(\incl, Γ_Δ)$-adjunction and base change we get a morphism
    \[
        (p_{2,Δ})_*p_{1,Δ}^! =
        (p₂)_* ∘ \incl ∘ Γ_Z ∘ p_1^! →
        (p₂)_* p₁^! \isoto
        f^!f_*.
    \]
    Combining this with $(f_*,f^!)$-adjunction we get the required map
    \[
        (p_{2,Δ})_*p_{1,Δ}^!(f^!\sheaf F) →
        f^!f_*(f^!\sheaf F) =
        f^!(f_*f^!)\sheaf F →
        f^!\sheaf F.
        \qedhere
    \]
\end{proof}

\printbibliography
\end{document}
