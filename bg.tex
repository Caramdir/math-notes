%type: notes
%title: $BG$
%tags: representation theory, stacks, principal bundles
\documentclass[english, no-theorem-numbers]{short-notes}

\usepackage{math-alg,math-ag,math-gl}

\addbibresource{math.bib}
%\bibliography{global.bib}


\title{\texorpdfstring{$BG$}{BG}}
\author{Clemens Koppensteiner}

\begin{document}

\maketitle

\begin{Def}
    Let $G$ be an algebraic group.
    Then $BG$ is the quotient stack $[\rquot{\pt}G]$.
\end{Def}

For any scheme $X$ the set $BG(X) = \Maps(X, BG)$ is the set of principal $G$-bundles on $X$.
The cover $\pt → BG$ corresponds to the trivial $G$-bundle on $\pt$.
Obviously the stack $BG$ is smooth.
In particular $\catIndCoh{BG} = \catQCoh{BG}$.

If $H \le G$ is a subgroup then there is a map $BH → BG$, given on $BH(X) → BG(X)$ as extension of $H$-bundles to $G$-bundles.

The following two lemmas are \cite[Exercise~1.21]{Heinloth:2010:LecturesModuliStackVectorBundlesOnCurve}
\begin{Lem}
    If $H$ is a closed subgroup of $G$, then 
    \[
        \pt ×_{BG} BH = \rquot GH.
    \]
\end{Lem}

\begin{Lem}
    If $H$ is a closed normal subgroup of $G$, then
    \[
        \pt ×_{B(G/H)} BG = BH.
    \]
\end{Lem}

Further useful notes (though in the topological setting) are \cite{Mitchell:unpublished:NotesOnPrincipalBundles}.

\section*{the case of a reductive group}

From now on let $G$ be reductive, $B$ a Borel and $H$ the corresponding torus.
We consider the correspondence 
\[
    BG \xleftarrow{p} BB \xrightarrow{q} BH.
\]

Since $G/B$ is a smooth projective variety, the map $p$ is proper, smooth and representable.
The map $p$ is smooth, but \emph{not} representable.

Thus the maps $p^*$, $p_*$ and $p^!$ are all defined, preserve coherence and give two pairs of adjoint functors $(p^*,p_*)$ and $(p_*,p^!)$.
The functors $p^*$ and $p^!$ differ (at most) by twisting with a line bundle.

Further $q^*$, $q_*$ and $q^!$ are defined and $q^*$ and $q^!$ differ by twisting with a line bundle.
The pair $(q^*, q_*)$ is adjoint.

\printbibliography
\end{document}
